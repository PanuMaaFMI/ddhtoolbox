% My choice: :
%-----------------
%\documentclass[a4paper,handout]{beamer}
%\documentclass[a4paper,presentation]{beamer}
\documentclass[aspectratio=169]{beamer}
%----------------------
%
%
%
%
%
%
%----------------------
% Yves Bouteloup :
%----------------------
%\documentclass[8pt,a4paper,presentation]{beamer}
%--------------------
% ENS templates:
%--------------------
%\documentclass{beamer}
%----------------------
%\documentclass[10pt,a4paper,presentation]{beamer}
%\documentclass[t,handout]{beamer}
%----------------------


%------------------------------
% pour windows (Font ecso0800 not found) :
%------------------------------
\usepackage{lmodern}

%-----------------------------
% The Frankfurt theme: (with )
%-----------------------------
\usetheme{Frankfurt}

%-----------------------
% Usefull packages:
%-----------------------
\usepackage{pgf}
\usepackage{bm}
%%\usepackage[french]{babel}
%%\usepackage[english,frenchb]{babel}
\usepackage[french,english]{babel} % prendre celui là ?

% Pour avoir la date et les titres en français :
%\usepackage[francais,english]{babel}
%\usepackage[latin1]{inputenc}
%\usepackage[T1]{fontenc}

%------------------------------
% pour avoir les equations :
%------------------------------
%\usepackage{sansmathaccent}
%\pdfmapfile{+sansmathaccent.map}

%----------------------------
% from ENS templates:
%----------------------------
\usepackage{pgfarrows,pgfnodes,pgfautomata,pgfheaps,pgfshade}
\usepackage{amsmath,amssymb}
%\usepackage[latin1]{inputenc}
\usepackage{colortbl} % ???
\usepackage{lipsum}
%----------------------------
% from Karim Yessad:
%----------------------------
%\usepackage[latin1]{inputenc}
%\usepackage[cyr]{aeguill}
%\usepackage[francais]{babel}

%----------------------------
% from Yves bouteloup:
%----------------------------
\usepackage[utf8]{inputenc} % !!! Important pour les accents français (Météo) !!!
\usepackage[T1]{fontenc} % !!! Important pour les accents français (Météo) !!!
%\usepackage{times} % pas certain...
\usepackage{amsfonts,amsbsy}
%\usepackage{euscript} % ??
%\usepackage{ulem} % ??
\usepackage{caption} % ??
%
\usepackage{tikz}
\usepackage{color,graphicx,subfigure}
\usepackage{epstopdf}
\usepackage{array}
\usepackage{verbatim}
%\usepackage[all]{xy}  % x-y pic (pour faire des dessins !)
\setbeamertemplate{sections/subsections in toc}[circle]
% Pour pouvoir superposer des choses (en particulier des légendes ...
\usepackage[absolute,showboxes,overlay]{textpos}  % déclaration du package
\textblockorigin{0pt}{0pt} % origine des positions (coin gauche) attention, axe y vers le bas
%\TPshowboxestrue          % affiche le contour des textblock
\TPshowboxesfalse          % n'affiche pas le contour des textblock

\DeclareMathAlphabet{\mathpzc}{OT1}{pzc}{m}{it}
%\newcommand{\lp}{\left(}
%\newcommand{\rp}{\right)}
%\newcommand{\lc}{\left[}
%\newcommand{\rc}{\right]}
%\newcommand{\la}{\left \{ }
%\newcommand{\ra}{\right \} }
\newcommand{\dr}{\partial}
\newcommand{\gray}[1]{\textcolor{gray}{#1}}
\newcommand{\red}[1]{\textcolor{red}{#1}}
\newcommand{\blue}[1]{\textcolor{blue}{#1}}
\newcommand{\mynode}[4]{\draw(#1,#2) node[right] (#3) {\phantom{AAAA}} node[right] {#4}}
% http://latexcolor.com/
\definecolor{mydarkred}{rgb}{0.5, 0.0, 0.0}
\definecolor{mydarkgreen}{rgb}{0.0, 0.5, 0.0}
\definecolor{mydarkblue}{rgb}{0.0, 0.0, 0.5}
\newcommand{\green}[1]{\textcolor{mydarkgreen}{#1}}

%-------------------
% from mf2016:
%-------------------
\definecolor{airmassmf1}{rgb}{1.000,0.941,0.686}
\definecolor{airmassmf2}{rgb}{0.768,0.756,0.690}
\definecolor{airmassmf3}{rgb}{0.568,0.607,0.682}
\definecolor{airmassmf4}{rgb}{0.384,0.482,0.655}
\definecolor{airmassmf5}{rgb}{0.227,0.388,0.623}
\definecolor{airmassmf6}{rgb}{0.086,0.313,0.588}
\definecolor{mftabname1}{rgb}{0.117,0.294,0.603}
\definecolor{mftabname2}{rgb}{0.941,0.208,0.184}
\definecolor{mftabedge1}{rgb}{0.768,0.800,0.906}
\definecolor{blockbg1}{rgb}{0.900,0.9,0.9}
\definecolor{mfblue}{rgb}{0.000,0.5607,0.7411}
%-------------------
% from mf2016:
%-------------------
%\setbeamercolor*{section in head/foot}{parent=palette tertiary}
%\setbeamercolor*{subsection in head/foot}{parent=palette primary}
%-------------------
% from mf2016:
%-------------------
% pour les pieds de page :
%\setbeamerfont{author in head/foot}{family=\sffamily}
%\setbeamerfont{title in head/foot}{family=\sffamily,series=\bfseries,size={\fontsize{6}{7}}}
%\setbeamerfont{date in head/foot}{family=\sffamily}
%\setbeamerfont{frame in head/foot}{family=\sffamily,series=\bfseries,size={\fontsize{6}{7}}}
%\setbeamerfont{section in head/foot}{size=\tiny,series=\normalfont}
%\setbeamerfont{subsection in head/foot}{family=\sffamily,size={\fontsize{1}{1}}}
%-------------------
% from mf2016: (style de couleur des item )
%-------------------
\setbeamercolor{itemize item}{fg=red} 
% ------------------------------------------------------------------------
% from mf2016: (Couleurs speciales)
% ------------------------------------------------------------------------
\definecolor{beamer@blendedblue}{rgb}{0.000,0.439,0.753}
\definecolor{mfgrey}{rgb}{0.250,0.250,0.250}
\definecolor{mfwhite}{rgb}{1.000,1.000,1.000}
\definecolor{mfblack}{rgb}{0.000,0.000,0.000}
\definecolor{mftitleblue}{rgb}{0.000,0.5607,0.7411}
\definecolor{mflogoblue}{rgb}{0.1215,0.3333,0.5843}
% Couleurs
%\setbeamercolor{frametitle}{fg=beamer@blendedblue}
%\setbeamercolor{alerted text}{fg=red}
%\setbeamercolor{block title}{use=structure,fg=white,bg=structure.fg!80!black}
%\setbeamercolor{block title alerted}{use=alerted text,fg=white,bg=alerted text.fg!80!black}
%\setbeamercolor{block body}{bg=mftabedge1}
\setbeamercolor{block body}{bg=blockbg1}
\setbeamercolor{block title example}{use=example text,fg=white,bg=example text.fg!80!black}
\setbeamercolor{block title}{fg=white,bg=mflogoblue}
\setbeamercolor{block title alerted}{use=alerted text,fg=white,bg=alerted text.fg!80!black}
\setbeamercolor{whitetext}{fg=white,bg=beamer@blendedblue!80!black}
\setbeamercolor{alerted text}{fg=red}
\setbeamercolor{frametitle}{fg=mfblue!70!mfblue, bg=white}
%\setbeamercolor{frametitle}{fg=mftitleblue}
\setbeamercolor{framesubtitle}{fg=mftitleblue}
%\setbeamercolor{block title}{use=structure,fg=white,bg=structure.fg!80!black}
%\setbeamercolor{block title alerted}{use=alerted text,fg=white,bg=alerted text.fg!80!black}
%\setbeamercolor{block title example}{use=example text,fg=white,bg=example text.fg!80!black}
%\setbeamercolor{whitetext}{fg=white,bg=beamer@blendedblue!80!black}

% ------------------------------------------------------------------------
% from mf2016: (Couleurs page de titre)
% ------------------------------------------------------------------------
%\setbeamercolor{title}{fg=mfblack}
%\setbeamercolor{subtitle}{fg=mfblack}
%\setbeamercolor{author}{fg=mftitleblue}
%\setbeamercolor{institute}{fg=mftitleblue}
%\setbeamercolor{date}{fg=mftitleblue}
% ------------------------------------------------------------------------
% from mf2016: (Polices d'ecriture)
% ------------------------------------------------------------------------
\setbeamerfont{frametitle}{family=\sffamily,series=\bfseries,size={\fontsize{13}{15}}}
\setbeamerfont{framesubtitle}{size=\normalsize}
\setbeamerfont{subsection in toc}{size=\small}
\setbeamerfont{subsubsection in toc}{size=\small}
% Polices d'ecriture page de titre
\setbeamerfont{title}{     family=\sffamily,series=\bfseries,size={\fontsize{13}{15}},parent=structure}
\setbeamerfont{subtitle}{  family=\sffamily,series=\bfseries,size=\normalsize,parent=title}
\setbeamerfont{author}{    family=\sffamily,series=\bfseries,size={\fontsize{10}{12}}}
\setbeamerfont{institute}{ family=\sffamily,series=\bfseries,size={\fontsize{10}{12}}}
\setbeamerfont{date}{      family=\sffamily,series=\bfseries,size={\fontsize{10}{12}}}

% ------------------------------------------------------------------------
% from mf2016:  (Blocs arrondis)
% ------------------------------------------------------------------------
%  \setbeamertemplate{blocks}[rounded][shadow=true,bg=yellow]
\newcommand{\bb}[1]{\begin{beamerboxesrounded}[shadow=true]{#1}}
\newcommand{\eb}{\end{beamerboxesrounded}}

% -----------------------
% from Y. Bouteloup:
% -----------------------
%\setbeamertemplate{sections/subsections in toc}[circle]
%
% Pour pouvoir superposer des choses (en particulier des légendes ...
%\usepackage[absolute,showboxes,overlay]{textpos}     % déclaration du package
%\textblockorigin{0pt}{0pt}                           %origine des positions (coin gauche) attention, axe y vers le bas !!
%\TPshowboxestrue                                  % affiche le contour des textblock
%\TPshowboxesfalse                                % n'affiche pas le contour des textblock


% ------------------------------------------------------------------------
%  from mf2016: Enlever la barre de navigation
% ------------------------------------------------------------------------
\setbeamertemplate{navigation symbols}{} 

% ------------------------------------------------------------------------
%  from mf2016: Affichage du titre
% ------------------------------------------------------------------------
%\setbeamertemplate{frametitle}
%  { %
%  \vskip0.1cm
%  \begin{beamercolorbox}[wd=\paperwidth,ht=1.1cm,leftskip=1.3cm]{frametitle}
%    \usebeamerfont{frametitle}\usebeamercolor[fg]{frametitle}\insertframetitle
%    %\vskip0.5em
%    \flushright \usebeamerfont{framesubtitle}\usebeamercolor[fg]{framesubtitle}{\itshape %\insertframesubtitle\hskip0.4cm}
%  \end{beamercolorbox}
%  \vskip0.1cm
%}
% \centering \usebeamerfont{framesubtitle}\usebeamercolor[fg]{framesubtitle}{\itshape \insertframesubtitle\hskip0.4cm}
% \flushright \usebeamerfont{framesubtitle}\usebeamercolor[fg]{framesubtitle}{\itshape \insertframesubtitle\hskip0.4cm}

% ------------------------------------------------------------------------
% from mf2016:  Reglage des listes et enumerations
% ------------------------------------------------------------------------
\setbeamertemplate{sections/subsections in toc}[square]
\setbeamertemplate{items}[default]
\setbeamertemplate{itemize item}[square]
\setbeamertemplate{itemize subitem}[circle]
\setbeamertemplate{itemize subsubitem}[triangle]
\setbeamercolor{itemize item}{fg=mftitleblue}
\setbeamercolor{itemize subitem}{fg=mfblack}
\setbeamercolor{itemize subsubitem}{fg=mfblack}

%-------------------
% from test04:       Navigation bar
%-------------------
\makeatletter
\def\insertnavigation#1{%
\vbox{{%
\usebeamerfont{section in head/foot}\usebeamercolor[fg]{section in head/foot}%
\beamer@xpos=0\relax%
\beamer@ypos=1\relax%
\hbox to #1{\hskip.3cm\setbox\beamer@sectionbox=\hbox{\kern1sp}%
  \ht\beamer@sectionbox=1.875ex%
  \dp\beamer@sectionbox=0.75ex%
    \hskip.3cm%
    \global\beamer@section@min@dim\z@
    \dohead%
    \beamer@section@set@min@width
  \box\beamer@sectionbox\hfill\hskip.3cm}%
}}}

%-------------------
% from test04:       Navigation bar / headline
%-------------------
%
\setbeamercolor{upper separation line head}{bg=mfblue!70!mfblue}
%
\setbeamertemplate{headline}
{%
\pgfuseshading{beamer@barshade}%
\ifbeamer@sb@subsection%
\vskip-9.75ex%
\else%
\vskip-7ex%
\fi%
\begin{beamercolorbox}[ignorebg,ht=2.25ex,dp=3.75ex]{section in head/foot}
\insertnavigation{\paperwidth}
\end{beamercolorbox}%
\ifbeamer@sb@subsection%
\begin{beamercolorbox}[ignorebg,ht=2.125ex,dp=1.125ex,%
  leftskip=.3cm,rightskip=.3cm plus1fil]{subsection in head/foot}
  \usebeamerfont{subsection in head/foot}\insertsubsectionhead
\end{beamercolorbox}%
\fi%
\begin{beamercolorbox}[colsep=1.5pt,ht=.75ex]{upper separation line head}
\end{beamercolorbox}
}%
\makeatother


%-----------------------------------
% from Ecrad (Y. Bouteloup):
%-----------------------------------
%\pgfdeclareimage[height=90mm,width=120mm]{garde}{cover_mf2016+cnrs+leffe}
%\pgfdeclareimage[height=96mm,width=128mm]{template}{diapo_type_mf2016+cnrs+leffe}

%-------------------
% from test04:
%-------------------
\setbeamercolor{section in head/foot}{fg=mflogoblue, bg=white}
%\setbeamercolor{section in head/foot}{fg=black, bg=white}
%
%\setbeamercolor{frametitle}{fg=red!70!black, bg=white}
%\setbeamercolor{upper separation line head}{bg=red!70!black}
%\setbeamercolor{section in head/foot}{fg=mfblue, bg=white}

\makeatletter
%------------------------------------------------------------------------------------
% from mf2016:  Pied de page (avec page et short-title, mais sans date)
%------------------------------------------------------------------------------------
%\defbeamertemplate*{footline}{Frankfurt theme}

%\setbeamertemplate{footline}
%{
%  \leavevmode%
%  \hbox{%
%  \begin{beamercolorbox}[wd=.20\paperwidth,ht=.3cm,dp=.3cm,left,leftskip=0.45cm]{frame in head/foot}%
%  %===== HERE ====================================================
%%    \usebeamerfont{section in head/foot} \usebeamercolor[mflogoblue]{section in head/foot} Page \insertframenumber / \inserttotalframenumber
%    \usebeamerfont{section in head/foot} \usebeamercolor[mflogoblue]{section in head/foot} Page \insertframenumber
%  %===== HERE ====================================================
%  \end{beamercolorbox}%
%  \begin{beamercolorbox}[wd=.60\paperwidth,ht=.3cm,dp=.15cm,center]{frame in head/foot}%
%  %===== HERE ====================================================
%    \usebeamerfont{section in head/foot} \usebeamercolor[mflogoblue]{section in head/foot} 
%    2-column model of conv. drafts accounting for the pert. pressure term
%  %===== HERE =====================================================
%  \end{beamercolorbox}%
%  }%
%  \vskip0pt%
%}


%===========================================

%-----------------------------------
% Pour les titre / sous-titre / date / afiliations / ... 
%-----------------------------------
%===== HERE ====== HERE ========
\title{
{\bf DDH: Diagnostics en Domaines Horizontaux} \\
\vspace{2mm}
{\bf Bilans dans ARPEGE et AROME }
}
%===== HERE ====== HERE ========
%\subtitle{{\it Réunion PROC/Phys (5 décembre 2017)}}
\author{
Fabrice Voitus et Jean-Marcel Piriou \\
\vspace{4mm}
{\it Météo-France / CNRM / GMAP} \\
\vspace{8mm}
{\it SISTEMD GMAP, Toulouse, 17 septembre 2019}
}
%===== HERE ====== HERE ========
%\institute{M\'et\'eo-France et CNRS-UMR3589 / Toulouse / France}
%\date{14 octobre 2017}
%\date{\today}
%\date{Réunion PROC/Phys (11 décembre 2017)}
\date{ }
%===== HERE ====== HERE ========

% \author[Hemaspaandra, Mukherji, Tantau]{%
%   Lane~Hemaspaandra\inst{1} \and Proshanto~Mukherji\inst{1} \and  Till~Tantau\inst{2}}
%  \institute[Universities of Rochester and Berlin]{
%    \inst{1}%
%    Department of Computer Science\\
%   University of Rochester
%    \and
%    \inst{2}%
%    Fakultät für Elektrotechnik und Informatik\\
%   Technical University of Berlin}

%============================================================================
\begin{document}


%============================================================================
\newcommand{\logos}[0]{
\begin{textblock*}{3cm}(13.4cm,8.0cm) 
\includegraphics[width=0.6cm]{logo_couleur_cnrm3_LR.jpg}
\hspace{0mm} 
\includegraphics[width=0.6cm]{LOGO_MTO1_bleu_new_BR.jpeg}
\hspace{0mm} 
\includegraphics[width=0.6cm]{logoCNRS_UMR3589_BR.jpeg}  
\end{textblock*}
}
%============================================================================

%----------------------------------------------
% Ajustement du comptage des cadres
%----------------------------------------------
\addtocounter{framenumber}{-1} % La page de garde est la page zéro

%===============================================================

%=============
 \begin{frame}[plain]
%=============
  \titlepage
%=============
\begin{textblock*}{6cm}(11.0cm,7.3cm)
\includegraphics[width=1.2cm]{logo_couleur_cnrm3.jpg}
\hspace{2mm}
\includegraphics[width=1.2cm]{LOGO_MTO1_bleu_new.jpeg}
\hspace{2mm}
\includegraphics[width=1.2cm]{logoCNRS_UMR3589.jpeg}
\end{textblock*}
\end{frame}

%
%-----------------------------------------------------------------------
% Choix de la fonte "times", plus lisible lorsque convertie en PDF.
%-----------------------------------------------------------------------
%
%\usepackage{times}
%
%-----------------------------------------------------------------------
% Package pour disposer des DOI dans la bibliographie.
%-----------------------------------------------------------------------
%
%\documentclass{article}
%\usepackage[utf8]{inputenc}
%\usepackage[backend=biber]{biblatex} 
%\usepackage[colorlinks]{hyperref}
%\ExecuteBibliographyOptions{doi=true}
%
%---------------------------------------------------------------------------
%
% Commandes communes aux deux langues et aux deux styles.
%
%---------------------------------------------------------------------------
%
% Caractères spéciaux.
%
% Début de paragraphe après ligne section ou chapitre.
%\newcommand{\pa}{}
\newcommand{\pa}{\noi}
%
% Début de paragraphe au sein d'une section ou chapitre.
%\newcommand{\p}{\vspace{0.4cm}}
\newcommand{\p}{\vspace{0.4cm}\noi}
\newcommand{\xx}{\p [...]}
\newcommand{\refp}[1]{(\ref{#1} p. \pageref{#1})}
\newcommand{\euros}{\textgreek{\euro}}
\newcommand{\lp}{\left(}
\newcommand{\rp}{\right)}
\newcommand{\lb}{\left\{}
\newcommand{\rb}{\right\}}
\newcommand{\lc}{\, \left[ \,}
\newcommand{\rc}{\, \right] \,}
\newcommand{\rd}{\, \right]}
\newcommand{\mb}{\null}
%\newcommand{\up}[1]{\raisebox{1ex}{\footnotesize#1}}
\newcommand{\dpdp}[2]{\frac{\partial #1}{\partial #2}}
\newcommand{\dd}[2]{\frac{d #1}{d #2}}
\newcommand{\ta}{{\theta}}
\newcommand{\tal}{{\theta_L}}
\newcommand{\tav}{{\theta_V}}
\newcommand{\tavl}{{\theta_{VL}}}
\newcommand{\tapw}{{\theta'_{w}}}
\newcommand{\tae}{{\theta_{E}}}
\newcommand{\taes}{{\theta_{ES}}}
\newcommand{\la}{\lambda}
\newcommand{\vp}{\varphi}
\newcommand{\ve}{\varepsilon}
\newcommand{\vt}{\vartheta}
%
\renewcommand{\ss}{\scriptstyle}
\newcommand{\sss}{\scriptscriptstyle}
\newcommand{\ds}{\displaystyle}
%
\newcommand{\noi}{\noindent}
\newcommand{\hs}{\hspace{1.5em}}
\newcommand{\ms}{\medskip}
%
\newcommand{\bey}{\begin{eqnarray}}
\newcommand{\eey}{\end{eqnarray}}
\newcommand{\bez}{\begin{eqnarray*}}
\newcommand{\eez}{\end{eqnarray*}}
\newcommand{\bay}{\begin{array}}
\newcommand{\eay}{\end{array}}
%
\newcommand{\ovl}[1]{\mkern1mu\overline{\mkern-1mu#1\mkern-1mu}\mkern1mu}
\newcommand{\ova}{\overrightarrow}
%
%-----------------------------------------------------------------------
% Ecrire du texte au sein d'une formule.
%-----------------------------------------------------------------------
%
\newcommand{\texte}[1]{\mathop{\rm #1}\nolimits}
%
% Trigonométrie et opérateurs.
\newcommand{\rot}{\mathop{\rm \overrightarrow{\vphantom{i}\rm ro}\mkern-.5mu t}\nolimits}
%\newcommand{\grad}{\mathop{\rm \overrightarrow{\vphantom{i}\rm gra}d}\nolimits}
\newcommand{\grad}{\vec{\nabla}}
\newcommand{\Grad}{\mathop{\rm \bf grad}\nolimits}
\newcommand{\Div}{\mathop{\rm div}\nolimits}
\renewcommand{\div}{\mathop{\rm div}\nolimits}
\newcommand{\Arctan}{\mathop{\rm Arctan}\nolimits}
\newcommand{\Arccos}{\mathop{\rm Arccos}\nolimits}
\newcommand{\Arcsin}{\mathop{\rm Arcsin}\nolimits}
\newcommand{\cotan}{\mathop{\rm cotan}\nolimits}
\newcommand{\ch}{\mathop{\rm ch}\nolimits}
\newcommand{\sh}{\mathop{\rm sh}\nolimits}
%\newcommand{\th}{\mathop{\rm th}\nolimits}
%
% Intégrales.
%\newcommand{\iint}{\int\!\!\!\int}
%\newcommand{\iiint}{\int\!\!\!\int\!\!\!\int}
%
% Date.
\newcommand{\jjmmaa}{\number\day.\number\month.\number\year}
%
% Fonctions texte.
\newcommand{\ie}{{\em i.e.}}
\newcommand{\cf}{{\em cf}}
%
% Fonctions mathématiques.
\newcommand{\AdvU}{(\vec{u}\cdot\vec{\nabla})}
\newcommand{\AdvV}{(\vec{v}\cdot\vec{\nabla})}
\newcommand{\DerDN}[3]{\frac{\ds d^{#3} #1}{\ds d #2^{#3}}}
\newcommand{\DerPN}[3]{\frac{\ds \partial^{#3} #1}{\ds \partial #2^{#3}}}
\newcommand{\DerP}[2]{\frac{\ds \partial #1}{\ds \partial #2}}
\newcommand{\DerSec}[2]{\frac{\ds {\partial}^{2} #1}{\ds \partial {#2}^{2}}}
\newcommand{\derx}[1]{\frac{\ds \partial #1}{\partial x}}
\newcommand{\dery}[1]{\frac{\ds \partial #1}{\partial y}}
\newcommand{\derz}[1]{\frac{\ds \partial #1}{\partial z}}
\newcommand{\dert}[1]{\frac{\ds \partial #1}{\partial t}}
\newcommand{\DerD}[2]{\frac{\ds d #1}{\ds d #2}}
\newcommand{\us}[1]{\frac{1}{#1}}
\newcommand{\VarGen}{\mu}
\newcommand{\ee}[1]{\cdot 10^{#1}}
%
% Unités.
\newcommand{\wm}{W\,m^{-2}}
%
% Fonctions ARPEGE.
\newcommand{\sbou}{{\sc Conv-Arpege-PNT}}
\newcommand{\sjef}{{\sc Conv-Arpege-Climat}}
\newcommand{\dpsg}{\frac{\ds dp}{\ds g}}
\newcommand{\mf}{{\sc Meteo-France}}
\newcommand{\ARP}{{\sc Arpege}}
\newcommand{\arp}{{\sc Arpege}}
\newcommand{\arpala}{{\sc Arpege-Aladin}}
\newcommand{\arpalaaro}{{\sc Arpege-Aladin-Arome}}
\newcommand{\alaro}{{\sc Alaro}}
\newcommand{\aro}{{\sc Arome}}
\newcommand{\ARO}{{\sc Arome}}
\newcommand{\arome}{{\sc Arome}}
\newcommand{\arptro}{{\sc Arpege-Tropiques}}
\newcommand{\mnh}{{\sc M\'eso-NH}}
\newcommand{\cnh}{{\sc COME-NH}}
\newcommand{\eme}{{\sc Emeraude}}
\newcommand{\ala}{{\sc Aladin}}
\newcommand{\ALA}{{\sc Aladin}}
\newcommand{\alanh}{{\sc Aladin-NH}}
\newcommand{\alae}{{\sc Aladin-Europe Centrale}}
\newcommand{\alaf}{{\sc Aladin-France}}
\newcommand{\alam}{{\sc Aladin-Maroc}}
\newcommand{\arppnt}{{\sc Arpege-PNT}}
\newcommand{\arpc}{{\sc Arpege-Climat}}
\newcommand{\ARPC}{{\sc Arpege-Climat}}
\newcommand{\arpifs}{{\sc Arpege-Ifs}}
\newcommand{\ifs}{{\sc Ifs}}
\newcommand{\arpa}{{\sc Arpege-Aladin}}
\newcommand{\mesonh}{{\sc Méso-NH}}
\newcommand{\cep}{{\sc CEP}}
\newcommand{\ecmwf}{{\sc Ecmwf}}
\newcommand{\grib}{{\sc GRIB}}
\newcommand{\GRIB}{{\sc GRIB}}
\newcommand{\re}{r_{\eta}}
\newcommand{\fp}[1]{F_{p#1}}
\newcommand{\cp}[1]{{c_{p}}_{#1}}
\newcommand{\Cp}[1]{{C_{p}}_{#1}}
\newcommand{\divi}[1]{\div_{#1}}
\newcommand{\interv}[2]{\lc #1,\,#2\rc}
\newcommand{\paire}[2]{\lb #1,\,#2\rb}
\newcommand{\ensemble}[2]{\lb #1,\,..., \, #2\rb}
%
\newcommand{\fpcl}[1]{F_{p#1}^{conv-l}}
\newcommand{\fpcn}[1]{F_{p#1}^{conv-n}}
\newcommand{\fpsl}[1]{F_{p#1}^{stra-l}}
\newcommand{\fpsn}[1]{F_{p#1}^{stra-n}}
\newcommand{\fpl}[1]{F_{p#1}^{l}}
\newcommand{\fpn}[1]{F_{p#1}^{n}}
\newcommand{\fccl}[1]{F_{c#1}^{conv-l}}
\newcommand{\fccn}[1]{F_{c#1}^{conv-n}}
\newcommand{\fcsl}[1]{F_{c#1}^{stra-l}}
\newcommand{\fcsn}[1]{F_{c#1}^{stra-n}}
\newcommand{\fcl}[1]{F_{c#1}^{l}}
\newcommand{\fcn}[1]{F_{c#1}^{n}}
\newcommand{\fc}[1]{F_{c#1}}
%
\newcommand{\fcptpcl}{F_{{c_p T}_{prec}}^{conv-l}}
\newcommand{\fcptpcn}{F_{{c_p T}_{prec}}^{conv-n}}
\newcommand{\fcptpsl}{F_{{c_p T}_{prec}}^{stra-l}}
\newcommand{\fcptpsn}{F_{{c_p T}_{prec}}^{stra-n}}
\newcommand{\fcptpl}{F_{{c_p T}_{prec}}^{l}}
\newcommand{\fcptpn}{F_{{c_p T}_{prec}}^{n}}
\newcommand{\fcptp}{F_{{c_p T}_{prec}}}
%
% Convection.
\newcommand{\dlnpb}{(\Delta\ln p)^{b}}
\newcommand{\dlnph}{(\Delta\ln p)^{h}}
\newcommand{\rbtm}{\tilde R_{b}^{-}}
\newcommand{\rbtp}{\tilde R_{b}^{+}}
\newcommand{\rvtp}{\tilde R_{v}^{+}}
\newcommand{\tdconv}[1]{\lp\DerP{#1}{t}\rp_{conv}}
\newcommand{\tdconvp}[1]{\lp\DerP{#1}{t}\rp_{conv\_\,prec}}
\newcommand{\fsd}{F_{s}^{dif\_\,tur}}
\newcommand{\fqd}{F_{q}^{dif\_\,tur}}
\newcommand{\fhd}{F_{h}^{dif\_\,tur}}
\newcommand{\tsd}{T_{s}^{dif\_\,tur}}
\newcommand{\tqd}{T_{q}^{dif\_\,tur}}
\newcommand{\thd}{T_{h}^{dif\_\,tur}}
\newcommand{\tvasc}{\vec{T}_{\vec{v}}^{conv}}
\newcommand{\omee}{\omega^{*}}
\newcommand{\omec}{\omega_c}
%
%-----------------------------------------------------------------------
% Commande servant à saisir l'essence d'un paragraphe,
% lorsqu'on en est, au cours de la rédaction, encore au stade
% de créer le plan détaillé.
%-----------------------------------------------------------------------
%
\newcommand{\ideepar}[1]{\p[{\bf Paragraphe: }#1]}
%
% Guillemets.
\def\og{\leavevmode\raise.3ex\hbox{$\scriptscriptstyle\langle\!\langle$\kern.05em}}
\def\fg{\leavevmode\unskip\kern.05em\raise.3ex\hbox{$\scriptscriptstyle\rangle\!\rangle$}}
%
% Commande "sujet" ouvrant un nouveau paragraphe séparé du précédent
% par une ligne horizontale et comportant un titre en gras.
\newcommand{\sujet}[1]{\noindent\rule{\textwidth}{0.1mm} {\bf \large #1 \newline }}
%
% Page de titre, 5 arguments:
%   1. Titre.
%   2. Sous-titre.
%   3. Auteur.
%   4. Version et/ou date.
%   5. Nom du fichier graphique à insérer.
%
\newcommand{\entete}[6]{
   {\pagestyle{empty}\null
   \begin{center}
   \begin{tabular}{c}
      \\[1ex] \huge \centerline{#1} \\[1ex]
      \huge #2 \\[2ex]
      \Large #3 \\[2ex] \Large #4 \\[1ex]
   \end{tabular}
   \end{center}
   \null\vspace{1cm}
  \begin{figure}[htbp]
    \centerline{
    \includegraphics
      [angle=0, 
      keepaspectratio=true,
      clip=true,#6]
      {#5}
    }
  \end{figure}
   \newpage\sommaire}}
%
%-----------------------------------------------------------------------
% Graphique commun pour latex et latex2html,
% d'après Ryad El Khatib le 26.5.2003.
%-----------------------------------------------------------------------
%
\newcommand{\graphryad}[5]
  {
  \begin{latexonly}
    \medskip\par
  \end{latexonly}
  \begin{figure}[!h]
    \label{fig:#1}
    \html{\htmlimage{align=center,transparent,antialias}}
    \html{\htmlborder{5}}
    \centering
    \html{\includegraphics[angle=#3]{#1.eps}}
    \latex{\framebox{\includegraphics[scale=#2,angle=#3]{#1.eps}}}
    \begin{latexonly}
      \caption[#4]{#5}
    \end{latexonly}
  \end{figure}
  \html{\begin{center}{\bf Figure~\ref{fig:#1}: }\emph{#5}\end{center}}
  }
%
%-----------------------------------------------------------------------
% Image graphicx en mode figure (ancienne version).
%-----------------------------------------------------------------------
%
\newcommand{\figp}[4]
  {
  \begin{figure}[htbp]
    \centerline{
      \includegraphics
        [angle=#3, 
        width=#2, 
        keepaspectratio=true,
        clip=true]
        {#1}
    }
    \caption{#4}
    \label{#1}
  \end{figure}
  }
%
%-----------------------------------------------------------------------
% Image graphicx en mode figure (nouvelle version).
% [h]: graphique ici.
% [b]: graphique en bas  de page.
% [t]: graphique en haut de page.
% [p]: graphique sur une page à part.
%-----------------------------------------------------------------------
%
\newcommand{\figpn}[6]
  {
  \begin{figure}[htbp]
    \centerline{
      \includegraphics
        [angle=#3, 
        width=#2, 
        keepaspectratio=true,
        clip=true]
        {#1}
    }
    \caption{ {\bf #4} {\em #5} {#6} }
    \label{#1}
  \end{figure}
  }
%
%-----------------------------------------------------------------------
% Image graphicx double. I.e. lit 2 fichiers EPS en entrée,
% et crée une seule figure, avec une seule légende.
% Un graphique est en haut, l'autre en bas.
% figdhb: FIGure Double, Haut-Bas.
%-----------------------------------------------------------------------
%
\newcommand{\figdhb}[7]
  {
  \begin{figure}[htbp]
    \centerline{
      \includegraphics
        [angle=#4, 
        width=#3, 
        keepaspectratio=true,
        clip=true]
        {#1}
    }
    \centerline{
      \includegraphics
        [angle=#4, 
        width=#3, 
        keepaspectratio=true,
        clip=true]
        {#2}
    }
    \caption{ {\bf #5} {\em #6} {#7} }
    \label{#1}
  \end{figure}
  }
%
%-----------------------------------------------------------------------
% Image graphicx double. I.e. lit 2 fichiers EPS en entrée,
% et crée une seule figure, avec une seule légende.
% Les 2 graphiques sont côte-à-côte.
% figdcc: FIGure Double, Côte à Côte.
%-----------------------------------------------------------------------
%
\newcommand{\figdcc}[7]
  {
  \begin{figure}[htbp]
    \centerline{
      \includegraphics
        [angle=#4, 
        width=#3, 
        keepaspectratio=true,
        clip=true]
        {#1}
      \includegraphics
        [angle=#4, 
        width=#3, 
        keepaspectratio=true,
        clip=true]
        {#2}
    }
    \caption{ {\bf #5} {\em #6} {#7} }
    \label{#1}
  \end{figure}
  }
%
%-----------------------------------------------------------------------
% Commande vide "£": elle sert seulement à définir une fonction
% qui ne fait rien sous latex, mais sera reconnue comme
% une chaîne de caractères par le coloriseur du source TEX.
% But: rendre des sources TEX plus lisibles, en pouvant coloriser
% des zones de texte particulières.
%-----------------------------------------------------------------------
%
%\newcommand{\£}{}
 % commandes mathématiques raccourcies, exemple: \DerP pour une dérivée partielle.


%%%%%%%%%%%%%%%%%%%%%%%%%%%%%%%%%%%%%%%%%%%%%%%%%%%%%%%%%%%%%%%%%%%%%%%%%%%%%
\section{Introduction}

%============================================================================
\subsection{Sommaire}
%============================================================================
\begin{frame}
%\frametitle{\vspace{0.5cm} \hspace{3.5cm} {\LARGE Sommaire} }
\frametitle{\vspace{0.5cm} \centerline{{\LARGE Sommaire}} }

\begin{textblock*}{13.5cm}(1.5cm,2.0cm)
\begin{columns}
\column{13.5cm}
\begin{block}<-1>{}
\begin{itemize}
% \Huge \huge \ LARGE \Large \large
% \normalsize \small \footnotesize \scriptsize \tiny
\item Motivation: pourquoi se servir des DDH ? Exemples d'usage.
\item Comment {\bf produire} des fichiers de DDH avec ARPEGE et AROME ? 
\item Développement des DDH flexibles.
\item Comment {\bf utiliser} les fichiers de DDH ? La ddhtoolbox.
\item Où trouver la documentation ? Qui contacter ?
\item Conclusions
%\item <3-> 
\end{itemize}
\end{block}
\end{columns}
\end{textblock*}

\logos

\end{frame}

%---------------------------------
% ajouter sur la slide suivante :
%---------------------------------
\addtocounter{framenumber}{-1}

%============================================================================
\subsection{Pourquoi les DDH ?}
\begin{frame}
\frametitle{\vspace{0.2cm} \centerline{\LARGE Pourquoi les DDH ?} }

\vspace{-1.0cm}
\begin{exampleblock}{\normalsize{Objectifs :}}
\begin{itemize}
 \item Etablir les équations de bilan des variables pronostiques du modèle sur un domaine défini, 
 afin d'améliorer la compréhension des interactions dynamiques et physiques du modèle. 
 $\Rightarrow$ contribuer au raffinement des parametrisations physiques.
 \item Fournir la valeur de champs diagnostiques (nébulosité, pression, etc).
\end{itemize}
\end{exampleblock}


\begin{block}{\normalsize{En sortie:}}
\begin{itemize}
 \item \textbf{Bilans} : masse (PP), vapeur d'eau (QV), énergie cinétique (KK), moment cinétique (A1,A2,A3), entropie (SS), énergie potentielle (EP), énergie thermique (CT), variables de surface, ... .
 \item \textbf{Domaines} : l'ensemble du domaine modèle, bandes zonales, domaines limités, et point de grille isolés. 
\end{itemize}
\end{block}
\end{frame}

%============================================================================

\subsection{Equation de bilan}
\begin{frame}
\frametitle{\normalsize Forme générale des équations de bilan}
\vspace{-0.25cm}
\begin{block}{\small Equation budget en coordonnée $\eta$ pour une variable $X$ :}
%  
\small
\begin{align*}
\frac{1}{g}\frac{\dr}{\dr t}\lp \frac{\dr \pi}{\dr\eta}X \rp & = 
\frac{1}{g}\frac{\dr \pi}{\dr\eta}\sum_{\mathpzc{D}} \mathcal{T}_{\mathpzc{D}} 
- \sum_{\varphi} \frac{\dr \mathcal{F}_{\varphi}}{\dr\eta}
+ \mathcal{E}
\end{align*}
%
\begin{itemize}
\item $\mathcal{T}_{\mathpzc{D}}$ : Tendances dynamiques sur les niveaux pleins du modèle, 
e.g advection horizontal et vertical, gradient de pression,  $\ldots$.
%
\item $\mathcal{F}_{\varphi}$ : Flux physiques verticaux calculés aux interfaces du modèle, 
e.g [rayonnement, convection, turbulence, precipitation, $\ldots$], issue des parametrizations.
%
\item $\mathcal{E}$ : Résidu numérique, non expliqué par la physique ou la dynamique, e.g diffusion horizontale, correction semi-implicite, ...]  
\end{itemize}
%
\end{block}
\end{frame}
%============================================================================
%============================================================================

\subsection{Equation d'énergie thermique}
\begin{frame}
\frametitle{{\normalsize Un petit exemple :} }
%\frametitle{\vspace{0.2cm} \centerline{\LARGE Pourquoi les DDH ?} }

\begin{block}{Bilan d'énergie thermique $c_p T$}
\footnotesize
\begin{align}
  \DerP{}{t}\lp\re c_p T\rp &=
  - \divi{\eta}\lp\re c_p T \vec{v}\rp
  - \DerP{}{\eta}\lp\re c_p T \dot{\eta}\rp
  + \re R T \frac{\omega}{p}\nonumber \\
  &  +\DerP{}{\eta}\lb F_{c_p T} + \fcptp{}
  + \fpl{} T \lc c_l - \cp{a} (1-\delta_m)\rc 
  + \fpn{} T \lc c_n - \cp{a} (1-\delta_m)\rc\rb\nonumber\\
  & 
  +\delta_m\fp{}\DerP{(\Phi+\frac{u^2+v^2}{2})}{\eta}
  -\vec v\cdot\DerP{\vec F_v^{\varphi}}{\eta}\label{Bilh}
\end{align}
\vspace*{0.5cm}\\
avec $r_{\eta}=-(1/g)\dr_{\eta}\pi$
\end{block}


\logos

\end{frame}

%============================================================================
\subsection{Bilan ARPEGE}
\begin{frame}
%\frametitle{\vspace{0.2cm} \hspace{3.5cm} {\LARGE Equations de bilan} }
\frametitle{\vspace{0.2cm} \centerline{\LARGE Equations de bilan} }

\begin{textblock*}{11.cm}(0.5cm,1.9cm)
\includegraphics<1->[width=11.cm]{zddhb_CT_ARPE_graph_doc_svg_HR.png}
\end{textblock*}

\begin{textblock*}{12.5cm}(1mm,1.3cm)
\begin{columns}
\column{11.5cm}
\begin{block}<-1>{}
\begin{center}
Bilan de température ARPEGE en moyenne horiz. globale
\end{center}
\end{block}
\end{columns}
\end{textblock*}

\logos

\end{frame}

%============================================================================
\subsection{Bilan AROME}
\begin{frame}
\frametitle{\vspace{0.2cm} \hspace{3.5cm} {\LARGE Equations de bilan } }

\begin{textblock*}{12.cm}(0.5cm,2.7cm)
\includegraphics<1->[width=12.cm]{yann.png}
\end{textblock*}

\begin{textblock*}{12.5cm}(1mm,1.3cm)
\begin{columns}
\column{11.5cm}
\begin{block}<-1>{}
\begin{center}
Bilan de température AROME sur la vallée de Chamonix. \\ Source Yann Seity.
\end{center}
\end{block}
\end{columns}
\end{textblock*}

\logos

\end{frame}

%============================================================================
\subsection{Circulations méridiennes}
\begin{frame}
%\frametitle{\vspace{0.2cm} \hspace{3.5cm} {\LARGE Circulations méridiennes} }
\frametitle{\vspace{0.2cm} \centerline{\LARGE Circulations méridiennes} }

\begin{textblock*}{2.7182818449cm}(-0.8cm,1.8cm)
\includegraphics<1->[width=18.cm]{DHFZOFCST+0072_xyuvc_1_doc_svg_HR.png}
\end{textblock*}


\end{frame}

%%%%%%%%%%%%%%%%%%%%%%%%%%%%%%%%%%%%%%%%%%%%%%%%%%%%%%%%%%%%%%%%%%%%%%%%%%%%%
\section{Production des DDH}


%============================================================================
\subsection{Planche d'annonce}
\begin{frame}
%\frametitle{\vspace{0.2cm} \hspace{3.5cm} {\LARGE Circulations méridiennes} }
\frametitle{\vspace{0.2cm} \centerline{\LARGE Production des DDH} }

\begin{textblock*}{2.7182818449cm}(0.8cm,1.8cm)
\includegraphics<1->[width=9.cm]{production.jpg}
\end{textblock*}

\begin{textblock*}{2.7182818449cm}(11cm,4.cm)
\includegraphics<1->[width=5.cm]{horloge-indus-a-rouages-effet-rouille-d125-500-13-29-138240_1.jpg}
\end{textblock*}

\begin{textblock*}{2.7182818449cm}(10.6cm,1.8cm)
\includegraphics<1->[width=5.5cm]{BullSequanaX-key-visual-538-303.jpg}
\end{textblock*}

\begin{textblock*}{2.7182818449cm}(1.cm,6cm)
\includegraphics<1->[width=1.1cm]{facile.png}
\end{textblock*}


\end{frame}


%============================================================================
\subsection{Histoire}
\begin{frame}
\frametitle{\normalsize Un peu d'Histoire }
\vspace{-.3cm}
\begin{block}{\normalsize{Développements des DDH dans nos modèles}}
\begin{enumerate}
 \onslide<1-2>\item 1991 : DDH initiés par Alain Joly pour ARPEGE $\Rightarrow$ 
 Stockage des champs DDH dans un tableau dédié PDHCV, 
 passé en argument et de dimension fixée au préalable (en dur)  
 $\Rightarrow$ \textbf{\textit{\green{thread-safe}}} mais \textbf{\textit{\red{non-flexible}}}.\\
 %
 \onslide<2-2>\item 2009 : DDH flexibles introduits par Olivier Rivière pour AROME  
 $\Rightarrow$ flexibilité via allocations dynamiques manuelles  
 d'un tableau de stockage global RDDH\_FIELD, passé par module \textbf{ddh\_mix} 
 $\Rightarrow$ \textbf{\textit{\green{flexible}}} mais \textbf{\textit{\red{non thread-safe}}}.\\
\end{enumerate}
\end{block}
\logos
\end{frame}
%============================================================================
\subsection{Ancien flux de données}
\begin{frame}
\frametitle{\small{\textbf{Ancien flux de données des DDH flexibles}}}
\vspace*{-0.5cm}
%
\begin{figure}[htbp]
	\centering
	\scriptsize
	\begin{tikzpicture}[x=6.5mm,y=-3mm,>=latex]
		\mynode{0}{0}{stepo}{stepo};
		\mynode{1}{1}{scan2m}{scan2m};
		\mynode{2}{2}{scan2h}{scan2h};
		\mynode{3}{3}{gp_model}{gp\_model};
		\mynode{4}{4}{cpg_drv}{cpg\_drv};
		\mynode{5}{5}{cpg}{cpg};
		\mynode{6}{6}{cpg_gp}{cpg\_gp};
		\draw(7,7) node[right] (cp_forcing) {cp\_forcing};
		\draw(10,7) node[right] (add_field_3d_1) {add\_field};
		\mynode{6}{8}{mf_phys}{mf\_phys};
		\draw(7,9) node[right] (aplpar) {aplpar};
		\draw(10,9) node[right] (add_field_3d_2) {add\_field};
		\draw(7,10) node[right] (cptend_new) {cptend\_new};
		\draw(10,10) node[right] (add_field_3d_3) {add\_field};
		\draw(7,11) node[right] (apl_arome)  {apl\_arome};
		\draw(10,11) node[right] (add_field_3d_4) {add\_field};
		\draw(7,12) node[right] (cputqy_arome) {cputqy\_arome};
		\draw(10,12) node[right] (add_field_3d_5) {add\_field};
		\mynode{6}{13}{cpg_dia}{cpg\_dia};
		\draw(7,14) node[right] (cpdyddh) {cpdyddh};
		\draw(10,14) node[right] (add_field_3d_6) {add\_field};	
		\draw(7,15) node[right] (cpcuddh1) {cpcuddh};
		\mynode{4}{16}{cpglag}{cpglag};
		\mynode{5}{17}{cpdyddhlag}{cpdyddhlag};
		\draw(6,18) node[right] (add_field_3d_7) {add\_field};
		\draw(6,19) node[right] (cpcuddh2) {cpcuddh};
		\mynode{1}{20}{posddh}{posddh};
		\mynode{2}{21}{ppsydh}{ppsydh};
		\mynode{2}{22}{ppfidh}{ppfidh};
		%
		\draw (stepo) |- (scan2m) |- (scan2h) |- (gp_model) |- (cpg_drv) |- (cpg);
		\draw (cpg) |- (cpg_gp) |- (cp_forcing) -- (add_field_3d_1);
		\draw (cpg) |- (mf_phys) |- (aplpar) -- (add_field_3d_2);
		\draw (mf_phys) |- (cptend_new) -- (add_field_3d_3);
		\draw (mf_phys) |- (apl_arome) -- (add_field_3d_4);
		\draw (mf_phys) |- (cputqy_arome) -- (add_field_3d_5);
		\draw (cpg) |- (cpg_dia) |- (cpdyddh) -- (add_field_3d_6);
		\draw (cpg_dia) |- (cpcuddh1);
		\draw (gp_model) |- (cpglag) |- (cpdyddhlag) |- (add_field_3d_7);
		\draw (cpdyddhlag) |- (cpcuddh2);
		\draw (stepo) |- (posddh) |- (ppfidh);
		\draw (posddh) |- (ppsydh);
		%		
		\draw[brown,line width=1pt] (4.5,4.5) rectangle (13,15.5);
		\draw[brown] (12,4.5) node[above left] {\bfseries OpenMP loop 1};
		\draw[brown,line width=1pt] (4.5,16.5) rectangle (13,19.5);
		\draw[brown] (12,19.5) node[below left] {\bfseries OpenMP loop 2};
		%
		\draw[blue] (15,10.5) node [right] (RDDH_FIELD1) {\bfseries RDDH\_FIELD};
		\draw[blue,->] (add_field_3d_1) -- (RDDH_FIELD1);
		\draw[blue,->] (add_field_3d_2) -- (RDDH_FIELD1);
		\draw[blue,->] (add_field_3d_3) -- (RDDH_FIELD1);
		\draw[blue,->] (add_field_3d_4) -- (RDDH_FIELD1);
		\draw[blue,->] (add_field_3d_5) -- (RDDH_FIELD1);
		\draw[blue,->] (add_field_3d_6) -- (RDDH_FIELD1);
		\draw[blue,->] (RDDH_FIELD1) -- +(0,3.5) -- (cpcuddh1);
		%
		\draw[blue] (15,18) node [right] (RDDH_FIELD2) {\bfseries RDDH\_FIELD};
		\draw[blue,->] (add_field_3d_7) -- (RDDH_FIELD2);
		\draw[blue,->] (RDDH_FIELD2)-- (cpcuddh2);		
		%
		\draw[green] (15,15) node[right] (HDCVB1) {\bfseries HDCVB*(:,:,ITHREAD)};
		\draw[green,->] (cpcuddh1) -- (HDCVB1);
		\draw[green] (15,19) node[right] (HDCVB2) {\bfseries HDCVB*(:,:,ITHREAD)};
		\draw[green,->] (cpcuddh2) -- (HDCVB2);
		\draw[green,->] (HDCVB1) -- (HDCVB2);
		\draw[green,->] (HDCVB2) -- +(0,2) -- (ppsydh);		
		%
		\draw[dashed] (14,0) -- (14,22);
		\draw (15,0) node[above right] {\bfseries GLOBAL VARIABLES};
	\end{tikzpicture}
\end{figure}
\end{frame}

%============================================================================
\subsection{Histoire suite}
\begin{frame}
\frametitle{\normalsize Un peu d'Histoire }
\vspace{-.3cm}
\begin{block}{\normalsize{Développements des DDH dans nos modèles}}
\begin{enumerate}
 \onslide<1-2>\item 1991 : DDH initiés par Alain Joly pour ARPEGE $\Rightarrow$ 
 Stockage des champs DDH dans un tableau dédié PDHCV, 
 passé en argument et de dimension fixée au préalable (en dur);  
 $\Rightarrow$ \textbf{\textit{\green{thread-safe}}} mais \textbf{\textit{\red{non-flexible}}}.\\
 %
 \onslide<1-2>\item 2009 : DDH flexibles introduits par Olivier Rivière pour AROME : 
 $\Rightarrow$ flexibilité via allocations dynamiques manuelles  
 d'un tableau de stockage global RDDH\_FIELD, passé par module \textbf{ddh\_mix} 
 $\Rightarrow$ \textbf{\textit{\green{flexible}}} mais \textbf{\textit{\red{non thread-safe}}}.\\
 % 
 \onslide<2-2>\item 2019 : DDH flexibles OpenMP à partir du cycle 46t1 $\Rightarrow$ allocations dynamiques automatiques 
 d'un pointeur de stockage, attribut d'une superstructure YDDH passée en argument dans chaque routine concernée $\Rightarrow$ \textbf{\textit{\green{flexible}}} et \textbf{\textit{\green{thread-safe}}}.
\end{enumerate}
\end{block}
\logos
\end{frame}
%============================================================================
\subsection{Nouveau flux de données}
\begin{frame}
\frametitle{\small{\textbf{Nouveau flux de données des DDH flexibles}}}
\vspace*{-0.7cm}
\begin{figure}[htbp]
	\centering
	\scriptsize
	\begin{tikzpicture}[x=7.mm,y=-2.6mm,>=latex]
		\mynode{0}{0}{stepo}{stepo};
		\mynode{1}{1}{scan2m}{scan2m};
		\mynode{2}{2}{scan2h}{scan2h};
		\mynode{3}{3}{gp_model}{gp\_model};
		\mynode{4}{4}{cpg_drv}{cpg\_drv(\blue{yddh})};
		\mynode{5}{5}{cpg}{cpg(\blue{yddh})};
		\mynode{6}{6}{setddh1}{\red{setddh}};
		\mynode{6}{7}{cpg_gp}{cpg\_gp(\blue{yddh})};
		\draw(7,8) node[right] (cp_forcing) {cp\_forcing};
		\draw(10,8) node[right] (add_field_1) {\red{new\_add\_field}};
		\mynode{6}{9}{mf_phys}{mf\_phys(\blue{yddh})};
		\draw(7,10) node[right] (aplpar) {aplpar};
		\draw(10,10) node[right] (add_field_2) {\red{new\_add\_field}};
		\draw(7,11) node[right] (cptend_new) {cptend\_new};
		\draw(10,11) node[right] (add_field_3) {\red{new\_add\_field}};
		\draw(7,12) node[right] (apl_arome)  {apl\_arome};
		\draw(10,12) node[right] (add_field_4) {\red{new\_add\_field}};
		\draw(7,13) node[right] (cputqy_arome) {cputqy\_arome};
		\draw(10,13) node[right] (add_field_5) {\red{new\_add\_field}};
		\mynode{6}{14}{cpg_dia}{cpg\_dia(\blue{yddh})};
		\draw(7,15) node[right] (cpdyddh) {cpdyddh};
		\draw(10,15) node[right] (add_field_6) {\red{new\_add\_field}};	
		\draw(7,16) node[right] (cpcuddh1) {cpcuddh};
		\mynode{6}{17}{cleanddh1}{\red{cleanddh}};
		\mynode{4}{18}{storeddh1}{\red{storeddh}};
		\mynode{4}{19}{cpglag}{cpglag(\blue{yddh})};
		\mynode{5}{20}{cpdyddhlag}{cpdyddhlag(\blue{yddh})};
		\draw(6,21) node[right] (setddh2) {\red{setddh}};
		\draw(6,22) node[right] (add_field_7) {\red{new\_add\_field}};
		\draw(6,23) node[right] (cpcuddh2) {cpcuddh};
		\draw(6,24) node[right] (cleanddh2) {\red{cleanddh}};
		\mynode{4}{25}{storeddh2}{\red{storeddh}};
		\mynode{1}{26}{posddh}{posddh};
		\mynode{2}{27}{ppsydh}{ppsydh};
		\mynode{2}{28}{ppfidh}{ppfidh};
		%
		\draw (stepo) |- (scan2m) |- (scan2h) |- (gp_model) |- (cpg_drv) |- (cpg);
		\draw (cpg) |- (setddh1);
		\draw (cpg) |- (cpg_gp) |- (cp_forcing) -- (add_field_1);
		\draw (cpg) |- (mf_phys) |- (aplpar) -- (add_field_2);
		\draw (mf_phys) |- (cptend_new) -- (add_field_3);
		\draw (mf_phys) |- (apl_arome) -- (add_field_4);
		\draw (mf_phys) |- (cputqy_arome) -- (add_field_5);
		\draw (cpg) |- (cpg_dia) |- (cpdyddh) -- (add_field_6);
		\draw (cpg_dia) |- (cpcuddh1);
		\draw (cpg) |- (cleanddh1);
		\draw (gp_model) |- (storeddh1);
		\draw (gp_model) |- (cpglag) |- (cpdyddhlag) |- (setddh2);
		\draw (cpdyddhlag) |- (add_field_7);
		\draw (cpdyddhlag) |- (cpcuddh2);
		\draw (cpdyddhlag) |- (cleanddh2);
		\draw (gp_model) |- (storeddh2);
		\draw (stepo) |- (posddh) |- (ppfidh);
		\draw (posddh) |- (ppsydh);
		%		
		\draw[brown,line width=1pt] (4.5,4.5) rectangle (13.5,17.5);
		\draw[brown] (12,4.75) node[above left] {\bfseries OpenMP loop 1};
		\draw[brown,line width=1pt] (4.5,19.5) rectangle (12,24.5);
		\draw[brown] (12,24.6) node[below left] {\bfseries OpenMP loop 2};
		%		
		%
		\draw[green] (15,15) node[right] (HDCVB1) {\bfseries HDCVB*(:,:,ITHREAD)};
		\draw[green,->] (cpcuddh1) -- (HDCVB1);
		\draw[green] (15,19) node[right] (HDCVB2) {\bfseries HDCVB*(:,:,ITHREAD)};
		\draw[green,->] (cpcuddh2) -- (HDCVB2);
		\draw[green,->] (HDCVB1) -- (HDCVB2);
		\draw[green,->] (HDCVB2) -- +(0,2) -- (ppsydh);		
		%
		\draw[dashed] (14,0) -- (14,22);
		\draw (15,0) node[above right] {\bfseries GLOBAL VARIABLES};
	\end{tikzpicture}
\end{figure}
\end{frame}
%
%============================================================================

\subsection{Récupération des tendances}

\begin{frame}[fragile]
\frametitle{\normalsize Récupération des tendances, flux et variables dans les DDH}
\vspace{-0.25cm}
\begin{alertblock}{} 
\footnotesize
\begin{itemize}
\item L'appel à NEW\_ADD\_FIELD\_3D ou NEW\_ADD\_FIELD\_2D via le module \textbf{ddh\_mix} 
n'importe où sous \textbf{cpg} ou \textbf{cpglag} permet de stocker au choix 
une variable, une tendance, ou un flux sous la forme d'un profil vertical 
dans les fichiers DDH.  
\end{itemize}
\end{alertblock}

\begin{exampleblock}{} 
\begin{footnotesize}
\begin{itemize}
\item Stockage de la variable X :
\begin{verbatim}
     ZX_VAR(:,:) = (PDELP(:,:)/RG)*PX(:,:)  
     CALL NEW_ADD_FIELD_3D(YMDDH,ZX_VAR,'VX_VAR',YDDH,CDTYPE='V')
\end{verbatim}
\item Stockage de Tendance pour le bilan DDH de X :
\begin{verbatim}
     ZXTEND(:,:) = (PDELP(:,:)/RG)*PXTEND(:,:)
     CALL NEW_ADD_FIELD_3D(YMDDH,ZXTEND,'TX_TEND',YDDH,CDTYPE='T')
\end{verbatim}
 \item Stockage de Flux pour le bilan DDH de X :
\begin{verbatim}
     CALL NEW_ADD_FIELD_3D(YMDDH,ZXFLUX,'FX_FLUX',YDDH,CDTYPE='F')
\end{verbatim}
\end{itemize}
\end{footnotesize}

\end{exampleblock}

\end{frame}
%============================================================================


%============================================================================
\subsection{Capuccino}

\begin{frame}[fragile]
\frametitle{{\normalsize Un exemple :} }
\vspace*{-0.25cm}
\begin{exampleblock}{}
\begin{tiny}
 \begin{verbatim} 
  SUBROUTINE DIAG_CAPUCCINO(KIDIA,KFDIA,KLEV,PINGREDIENTS,PDOSAGE,YDLDDH,YDMDDH,YDDDH)
  
  USE DDH_MIX, ONLY  : TYP_DDH,NEW_ADD_FIELD_3D
  USE YOMMDDH, ONLY  : TMDDH
  USE YOMLDDH, ONLY  : TLDDH     
  ....
  TYPE(TLDDH)       ,INTENT(IN)    :: YDLDDH ! ddh useful logicals
  TYPE(TMDDH)       ,INTENT(INOUT) :: YDMDDH ! ddh useful metadata 
  TYPE(TYP_DDH)     ,INTENT(INOUT) :: YDDDH  ! ddh superstructure  
  ....
  DO JLEV=1,KLEV
     DO JLON = KIDIA,KFDIA      
        ZCOFFEE(JLON,JLEV)      = .... 
        ZMILK(JLON,JLEV)        = ....
        ZCHOCOLATE(JLON,JLEV)   = ....
     ENDDO
  ENDDO   
  ....
  IF (YDLDDH%LFLEXDIA).AND.(YDLDDH%LDDH_OMP) THEN
     CALL NEW_ADD_FIELD_3D(YDMDDH,ZCOFFEE,'FCO_COFFEE',YDDDH)
     CALL NEW_ADD_FIELD_3D(YDMDDH,ZMILK,'CO_MILK',YDDDH,CDTYPE='F')
     CALL NEW_ADD_FIELD_3D(YDMDDH,ZCHOCOLATE,'FCO_CHOCO',YDDDH)
  ENDIF
  ....
  END  SUBROUTINE DIAG_CAPUCCINO
  \end{verbatim}  
  \end{tiny}
\end{exampleblock}
\end{frame}

%============================================================================
\subsection{Namelist utilisateur}

\begin{frame}[fragile]
\frametitle{{\normalsize Namelist utilisateur: un exemple} }
\vspace*{-0.4cm}
\begin{exampleblock}{}
\begin{footnotesize}
\begin{verbatim} 
 &NAMDDH                                                                                            
  BDEDDH(1,1)=4., ! domaine : un seul point de grille.
  BDEDDH(2,1)=1., ! plan virtuel.
  BDEDDH(3,1)=1.374, ! longitude du point.
  BDEDDH(4,1)=43.575, ! latitude du point.

  BDEDDH(1,2)=3., ! domaine : rectangle en lat-lon.
  BDEDDH(2,2)=1., ! plan virtuel.
  BDEDDH(3,2)=0.0, ! coin 1: longitude.
  BDEDDH(4,2)=45.0, ! coin 1: latitude.
  BDEDDH(5,2)=0.1, ! coin 2: longitude.
  BDEDDH(6,2)=44.8, ! coin 2: latitude.

  LHDDOP=.TRUE., ! activation domaines limités.
  LHDEFD=.TRUE., ! activation sortie fichiers domaines limités.
  LHDHKS=.TRUE., ! activation champs standard.
  LFLEXDIA=.TRUE., ! activation des DDH flexibles.
  LDDH_OMP=.TRUE., ! activation des DDH thread-safe.
 /
\end{verbatim}  
\end{footnotesize}
\end{exampleblock}
\end{frame}
%%%%%%%%%%%%%%%%%%%%%%%%%%%%%%%%%%%%%%%%%%%%%%%%%%%%%%%%%%%%%%%%%%%%%%%%%%%%%
\section{Utilisation des DDH}

%============================================================================
\subsection{Planche d'annonce}
\begin{frame}
%\frametitle{\vspace{0.2cm} \hspace{3.5cm} {\LARGE Circulations méridiennes} }
\frametitle{\vspace{0.2cm} \centerline{\LARGE Utilisation des DDH} }

\begin{textblock*}{2.7182818449cm}(0.8cm,1.8cm)
\includegraphics<1->[width=12.cm]{enfants.png}
\end{textblock*}

\logos

\end{frame}
%============================================================================
\subsection{ddhtoolbox}
%============================================================================
\begin{frame}
%\frametitle{\vspace{0.5cm} \hspace{3.5cm} {\LARGE Sommaire} }
\frametitle{\vspace{0.3cm} \centerline{{\LARGE ddhtoolbox}} }

\begin{textblock*}{11.cm}(1cm,1.5cm)
\includegraphics<1->[width=11.cm]{synoptic.png}
\end{textblock*}

\logos

\end{frame}


%============================================================================
\subsection{Emagramme}
%============================================================================
\begin{frame}
%\frametitle{\vspace{0.5cm} \hspace{3.5cm} {\LARGE Sommaire} }
\frametitle{\vspace{0.3cm} \centerline{{\LARGE Tracés type émagramme}} }

\begin{textblock*}{2.7182818449cm}(0.3cm,2.5cm)
\includegraphics<1->[width=8.cm]{theta_toulouse.png}
\includegraphics<1->[width=8.cm]{theta_tropiques.png}
\end{textblock*}


\begin{textblock*}{13.5cm}(1.5cm,1.5cm)
\begin{columns}
\column{13.5cm}
\begin{block}<2->{}
\begin{itemize}
% \Huge \huge \ LARGE \Large \large
% \normalsize \small \footnotesize \scriptsize \tiny
\item get hendrix:/home/m/mxpt/mxpt001/vortex/arpege/4dvarfr/OPER/2019/09/17 \\
/T0000P/forecast/ddh.arpege-zonal.tl1798-c22.ddhpack.tgz 
\item tar tvf
\item ddhr -ddd DHFZOFCST+0072 \textbf{\green{renseignements en clair sur chaque domaine}}
\item ddht -cEXTRAIT\_DOMAIN -E13-18 -1DHFZOFCST+0018 -sDHFZOFCST+0018.Tropiques.lfa \textbf{\green{extraction de domaines}}
\item ddhmh DHFZOFCST+0018.Tropiques.lfa \textbf{\green{moyenne horizontale}}
\item ddh2scm DHFZOFCST+0018.Tropiques.lfa.mh \textbf{\green{conversion en fichier type modèle 1D}}
\item ms DHFZOFCST+0018.Tropiques.lfa.mh.Dom001.Var\_fin.scm \textbf{\green{model sounding}}
%\item <3-> 
\end{itemize}
\end{block}
\end{columns}
\end{textblock*}



\end{frame}


%============================================================================
\subsection{Bilan de T d'une après-midi à Toulouse}
%============================================================================
\begin{frame}
%\frametitle{\vspace{0.5cm} \hspace{3.5cm} {\LARGE Sommaire} }
\frametitle{\vspace{0.3cm} \centerline{{\LARGE Bilan de température d'une après-midi à Toulouse}} }

\begin{textblock*}{2.7182818449cm}(1.3cm,1.7cm)
\includegraphics<1->[width=11.cm]{bil_CT_Toulouse.png}
\end{textblock*}


\begin{textblock*}{13.5cm}(1.5cm,1.5cm)
\begin{columns}
\column{13.5cm}
\begin{block}<2->{}
\begin{itemize}
% \Huge \huge \ LARGE \Large \large
% \normalsize \small \footnotesize \scriptsize \tiny
\item get hendrix:/home/m/mxpt/mxpt001/vortex/arpege/4dvarfr/OPER/2019/09/17 \\
/T0000P/forecast/ddh.arpege-dlimited.tl1798-c22.ddhpack.tgz
\item tar tvf
\item fddh="DHFDLFCST+0018.lfa"
\item ddhr -ddd \$fddh \textbf{\green{renseignements en clair sur chaque domaine}}
\item ddht -cEXTRAIT\_DOMAIN -E10 -1\$fddh -s\$fddh.10.lfa \textbf{\green{extraction du domaine Toulouse}}
\item ddht -cDIFFE\_EC2\_EC1 -2DHFDLFCST+0018.lfa.10.lfa -1DHFDLFCST+0012.lfa.10.lfa -sDHFDLFCST+18-12.lfa.10.lfa \textbf{\green{différence entre 12h et 18h}}
\item ddhb -v oper/CT -i DHFDLFCST+18-12.lfa.10.lfa \textbf{\green{bilan de température entre 12h et 18h}}
%\item <3-> 
\end{itemize}
\end{block}
\end{columns}
\end{textblock*}


\end{frame}
%%%%%%%%%%%%%%%%%%%%%%%%%%%%%%%%%%%%%%%%%%%%%%%%%%%%%%%%%%%%%%%%%%%%%%%%%%%%%
\section{Documentation des DDH}

%============================================================================
\subsection{Documentation des DDH}
\begin{frame}
\frametitle{\vspace{0.2cm} \centerline{\LARGE Documentation des DDH} }

\vspace{-1.0cm}
\begin{exampleblock}{} 
\begin{itemize}
\item {\bf Documentation complète}: namelists, équations, production, ddhtoolbox; sur GmapDoc : http://www.umr-cnrm.fr/gmapdoc/IMG/pdf/ddh.pdf
\item {\bf Présent exposé} : sur Intranet : http://intra.cnrm.meteo.fr/gmap-proc/perso/jean-marcel.piriou/cours\_memos\_docs/ddh\_presentation\_introductive.pdf
\end{itemize}
\end{exampleblock}

\begin{block}<2->{\normalsize{Hotline de niveau exceptionnel :) }}
\begin{itemize}
 \item {\bf Fabrice Voitus (5-85 77)} pour la production de fichiers de DDH dans AROME et ARPEGE.
 \item {\bf Jean-Marcel Piriou (5-84 62)} pour la ddhtoolbox et l'utilisation des DDH sur PC. 
\end{itemize}
\end{block}

\logos 

\end{frame}


%%%%%%%%%%%%%%%%%%%%%%%%%%%%%%%%%%%%%%%%%%%%%%%%%%%%%%%%%%%%%%%%%%%%%%%%%%%%%
\section{Conclusion}


%============================================================================
\subsection{Conclusion}
\begin{frame}
\frametitle{\vspace{0.2cm} \centerline{\LARGE Conclusion} }

\vspace{-1.0cm}
\begin{exampleblock}{\normalsize{Objectifs :}}
\begin{itemize}
 \item Etablir les équations de bilan des variables pronostiques du modèle sur un domaine défini, 
 afin d'améliorer la compréhension des interactions dynamiques et physiques du modèle. 
 $\Rightarrow$ contribuer au raffinement des parametrisations physiques.
 \item Fournir la valeur de champs diagnostiques (nébulosité, pression, etc).
\end{itemize}
\end{exampleblock}

\begin{block}{\normalsize{Moyens:}}
\begin{itemize}
 \item \textbf{Bilans} : masse (PP), vapeur d'eau (QV), énergie cinétique (KK), moment cinétique (A1,A2,A3), entropie (SS), énergie potentielle (EP), énergie thermique (CT), variables de surface, ... .
 \item \textbf{Domaines} : l'ensemble du domaine modèle, bandes zonales, domaines limités, et point de grille isolés. 
 \item \textbf{Codage à la volée} de diagnostics nouveaux via NEW\_ADD\_FIELD\_2D et NEW\_ADD\_FIELD\_3D.
\end{itemize}
\end{block}
\end{frame}

%============================================================================
\end{document}
