\documentclass[10pt,french]{book}
\usepackage[T1]{fontenc}
\usepackage[utf8]{inputenc}
\usepackage{geometry}
\geometry{verbose,tmargin=2cm,bmargin=2cm,lmargin=2cm,rmargin=2cm}
\usepackage[round]{natbib}
\usepackage{color}
\usepackage{textcomp}
\usepackage{url}
\usepackage{graphicx}
\usepackage{babel}
\usepackage[colorlinks]{hyperref}
\addto\extrasfrench{%
   \providecommand{\og}{\leavevmode\flqq~}
   \providecommand{\fg}{\ifdim\lastskip>\z@\unskip\fi~\frqq}
}
%
%-----------------------------------------------------------------------
% Commandes perso.
%-----------------------------------------------------------------------
%
%---------------------------------------------------------------------------
%
% Commandes particulières au Français,
% communes aux deux styles.
%
%---------------------------------------------------------------------------
%
% Permet de taper \^i pour avoir \^\i 
%\def\^#1{\if#1i{\accent"5E\i}\else{\accent"5E #1}\fi}%
%
% Permet de taper \"i pour avoir \"\i 
%\def\"#1{\if#1i{\accent"7F\i}\else{\accent"7F #1}\fi}%
%
% En-ete X à Y...
\newcommand{\XaY}[3]{\noindent {\bf #1}\hfill le \jjmmaa.\vspace{0.5cm}\newline
	à\vspace{0.5cm}\newline {\bf #2}\vspace{1.1cm}\newline
	\fbox{\hspace{0.5cm} \Large{\bf Objet:} #3 \hspace{0.5cm}}
	\vspace{1.1cm}\newline}
%
% Date dépendant du choix de la langue.
%
\newcommand{\aujour}{\number\day\space \ifcase\month\or janvier
\or février\or mars\or avril\or mai\or juin\or juillet
\or août\or septembre\or octobre\or novembre\or décembre\fi
\space\number\year}
%\newcommand{\aujour}{\today}
%
% Sommaire doit pointer sur tableofcontents.
\newcommand{\sommaire}{\tableofcontents}
%
%-----------------------------------------------------------------------
% Remerciements.
%-----------------------------------------------------------------------
%
\newcommand{\acknowledgement}{\p {\bf Remerciements}}
%
%-----------------------------------------------------------------------
% Titre type AMS.
%-----------------------------------------------------------------------
%
\newcommand{\amstitle}{\maketitle}

%
%-----------------------------------------------------------------------
% Choix de la fonte "times", plus lisible lorsque convertie en PDF.
%-----------------------------------------------------------------------
%
%\usepackage{times}
%
%-----------------------------------------------------------------------
% Package pour disposer des DOI dans la bibliographie.
%-----------------------------------------------------------------------
%
%\documentclass{article}
%\usepackage[utf8]{inputenc}
%\usepackage[backend=biber]{biblatex} 
%\usepackage[colorlinks]{hyperref}
%\ExecuteBibliographyOptions{doi=true}
%
%---------------------------------------------------------------------------
%
% Commandes communes aux deux langues et aux deux styles.
%
%---------------------------------------------------------------------------
%
% Caractères spéciaux.
%
% Début de paragraphe après ligne section ou chapitre.
%\newcommand{\pa}{}
\newcommand{\pa}{\noi}
%
% Début de paragraphe au sein d'une section ou chapitre.
%\newcommand{\p}{\vspace{0.4cm}}
\newcommand{\p}{\vspace{0.4cm}\noi}
\newcommand{\xx}{\p [...]}
\newcommand{\refp}[1]{(\ref{#1} p. \pageref{#1})}
\newcommand{\euros}{\textgreek{\euro}}
\newcommand{\lp}{\left(}
\newcommand{\rp}{\right)}
\newcommand{\lb}{\left\{}
\newcommand{\rb}{\right\}}
\newcommand{\lc}{\, \left[ \,}
\newcommand{\rc}{\, \right] \,}
\newcommand{\rd}{\, \right]}
\newcommand{\mb}{\null}
%\newcommand{\up}[1]{\raisebox{1ex}{\footnotesize#1}}
\newcommand{\dpdp}[2]{\frac{\partial #1}{\partial #2}}
\newcommand{\dd}[2]{\frac{d #1}{d #2}}
\newcommand{\ta}{{\theta}}
\newcommand{\tal}{{\theta_L}}
\newcommand{\tav}{{\theta_V}}
\newcommand{\tavl}{{\theta_{VL}}}
\newcommand{\tapw}{{\theta'_{w}}}
\newcommand{\tae}{{\theta_{E}}}
\newcommand{\taes}{{\theta_{ES}}}
\newcommand{\la}{\lambda}
\newcommand{\vp}{\varphi}
\newcommand{\ve}{\varepsilon}
\newcommand{\vt}{\vartheta}
%
\renewcommand{\ss}{\scriptstyle}
\newcommand{\sss}{\scriptscriptstyle}
\newcommand{\ds}{\displaystyle}
%
\newcommand{\noi}{\noindent}
\newcommand{\hs}{\hspace{1.5em}}
\newcommand{\ms}{\medskip}
%
\newcommand{\bey}{\begin{eqnarray}}
\newcommand{\eey}{\end{eqnarray}}
\newcommand{\bez}{\begin{eqnarray*}}
\newcommand{\eez}{\end{eqnarray*}}
\newcommand{\bay}{\begin{array}}
\newcommand{\eay}{\end{array}}
%
\newcommand{\ovl}[1]{\mkern1mu\overline{\mkern-1mu#1\mkern-1mu}\mkern1mu}
\newcommand{\ova}{\overrightarrow}
%
%-----------------------------------------------------------------------
% Ecrire du texte au sein d'une formule.
%-----------------------------------------------------------------------
%
\newcommand{\texte}[1]{\mathop{\rm #1}\nolimits}
%
% Trigonométrie et opérateurs.
\newcommand{\rot}{\mathop{\rm \overrightarrow{\vphantom{i}\rm ro}\mkern-.5mu t}\nolimits}
%\newcommand{\grad}{\mathop{\rm \overrightarrow{\vphantom{i}\rm gra}d}\nolimits}
\newcommand{\grad}{\vec{\nabla}}
\newcommand{\Grad}{\mathop{\rm \bf grad}\nolimits}
\newcommand{\Div}{\mathop{\rm div}\nolimits}
\renewcommand{\div}{\mathop{\rm div}\nolimits}
\newcommand{\Arctan}{\mathop{\rm Arctan}\nolimits}
\newcommand{\Arccos}{\mathop{\rm Arccos}\nolimits}
\newcommand{\Arcsin}{\mathop{\rm Arcsin}\nolimits}
\newcommand{\cotan}{\mathop{\rm cotan}\nolimits}
\newcommand{\ch}{\mathop{\rm ch}\nolimits}
\newcommand{\sh}{\mathop{\rm sh}\nolimits}
%\newcommand{\th}{\mathop{\rm th}\nolimits}
%
% Intégrales.
%\newcommand{\iint}{\int\!\!\!\int}
%\newcommand{\iiint}{\int\!\!\!\int\!\!\!\int}
%
% Date.
\newcommand{\jjmmaa}{\number\day.\number\month.\number\year}
%
% Fonctions texte.
\newcommand{\ie}{{\em i.e.}}
\newcommand{\cf}{{\em cf}}
%
% Fonctions mathématiques.
\newcommand{\AdvU}{(\vec{u}\cdot\vec{\nabla})}
\newcommand{\AdvV}{(\vec{v}\cdot\vec{\nabla})}
\newcommand{\DerDN}[3]{\frac{\ds d^{#3} #1}{\ds d #2^{#3}}}
\newcommand{\DerPN}[3]{\frac{\ds \partial^{#3} #1}{\ds \partial #2^{#3}}}
\newcommand{\DerP}[2]{\frac{\ds \partial #1}{\ds \partial #2}}
\newcommand{\DerSec}[2]{\frac{\ds {\partial}^{2} #1}{\ds \partial {#2}^{2}}}
\newcommand{\derx}[1]{\frac{\ds \partial #1}{\partial x}}
\newcommand{\dery}[1]{\frac{\ds \partial #1}{\partial y}}
\newcommand{\derz}[1]{\frac{\ds \partial #1}{\partial z}}
\newcommand{\dert}[1]{\frac{\ds \partial #1}{\partial t}}
\newcommand{\DerD}[2]{\frac{\ds d #1}{\ds d #2}}
\newcommand{\us}[1]{\frac{1}{#1}}
\newcommand{\VarGen}{\mu}
\newcommand{\ee}[1]{\cdot 10^{#1}}
%
% Unités.
\newcommand{\wm}{W\,m^{-2}}
%
% Fonctions ARPEGE.
\newcommand{\sbou}{{\sc Conv-Arpege-PNT}}
\newcommand{\sjef}{{\sc Conv-Arpege-Climat}}
\newcommand{\dpsg}{\frac{\ds dp}{\ds g}}
\newcommand{\mf}{{\sc Meteo-France}}
\newcommand{\ARP}{{\sc Arpege}}
\newcommand{\arp}{{\sc Arpege}}
\newcommand{\arpala}{{\sc Arpege-Aladin}}
\newcommand{\arpalaaro}{{\sc Arpege-Aladin-Arome}}
\newcommand{\alaro}{{\sc Alaro}}
\newcommand{\aro}{{\sc Arome}}
\newcommand{\ARO}{{\sc Arome}}
\newcommand{\arome}{{\sc Arome}}
\newcommand{\arptro}{{\sc Arpege-Tropiques}}
\newcommand{\mnh}{{\sc M\'eso-NH}}
\newcommand{\cnh}{{\sc COME-NH}}
\newcommand{\eme}{{\sc Emeraude}}
\newcommand{\ala}{{\sc Aladin}}
\newcommand{\ALA}{{\sc Aladin}}
\newcommand{\alanh}{{\sc Aladin-NH}}
\newcommand{\alae}{{\sc Aladin-Europe Centrale}}
\newcommand{\alaf}{{\sc Aladin-France}}
\newcommand{\alam}{{\sc Aladin-Maroc}}
\newcommand{\arppnt}{{\sc Arpege-PNT}}
\newcommand{\arpc}{{\sc Arpege-Climat}}
\newcommand{\ARPC}{{\sc Arpege-Climat}}
\newcommand{\arpifs}{{\sc Arpege-Ifs}}
\newcommand{\ifs}{{\sc Ifs}}
\newcommand{\arpa}{{\sc Arpege-Aladin}}
\newcommand{\mesonh}{{\sc Méso-NH}}
\newcommand{\cep}{{\sc CEP}}
\newcommand{\ecmwf}{{\sc Ecmwf}}
\newcommand{\grib}{{\sc GRIB}}
\newcommand{\GRIB}{{\sc GRIB}}
\newcommand{\re}{r_{\eta}}
\newcommand{\fp}[1]{F_{p#1}}
\newcommand{\cp}[1]{{c_{p}}_{#1}}
\newcommand{\Cp}[1]{{C_{p}}_{#1}}
\newcommand{\divi}[1]{\div_{#1}}
\newcommand{\interv}[2]{\lc #1,\,#2\rc}
\newcommand{\paire}[2]{\lb #1,\,#2\rb}
\newcommand{\ensemble}[2]{\lb #1,\,..., \, #2\rb}
%
\newcommand{\fpcl}[1]{F_{p#1}^{conv-l}}
\newcommand{\fpcn}[1]{F_{p#1}^{conv-n}}
\newcommand{\fpsl}[1]{F_{p#1}^{stra-l}}
\newcommand{\fpsn}[1]{F_{p#1}^{stra-n}}
\newcommand{\fpl}[1]{F_{p#1}^{l}}
\newcommand{\fpn}[1]{F_{p#1}^{n}}
\newcommand{\fccl}[1]{F_{c#1}^{conv-l}}
\newcommand{\fccn}[1]{F_{c#1}^{conv-n}}
\newcommand{\fcsl}[1]{F_{c#1}^{stra-l}}
\newcommand{\fcsn}[1]{F_{c#1}^{stra-n}}
\newcommand{\fcl}[1]{F_{c#1}^{l}}
\newcommand{\fcn}[1]{F_{c#1}^{n}}
\newcommand{\fc}[1]{F_{c#1}}
%
\newcommand{\fcptpcl}{F_{{c_p T}_{prec}}^{conv-l}}
\newcommand{\fcptpcn}{F_{{c_p T}_{prec}}^{conv-n}}
\newcommand{\fcptpsl}{F_{{c_p T}_{prec}}^{stra-l}}
\newcommand{\fcptpsn}{F_{{c_p T}_{prec}}^{stra-n}}
\newcommand{\fcptpl}{F_{{c_p T}_{prec}}^{l}}
\newcommand{\fcptpn}{F_{{c_p T}_{prec}}^{n}}
\newcommand{\fcptp}{F_{{c_p T}_{prec}}}
%
% Convection.
\newcommand{\dlnpb}{(\Delta\ln p)^{b}}
\newcommand{\dlnph}{(\Delta\ln p)^{h}}
\newcommand{\rbtm}{\tilde R_{b}^{-}}
\newcommand{\rbtp}{\tilde R_{b}^{+}}
\newcommand{\rvtp}{\tilde R_{v}^{+}}
\newcommand{\tdconv}[1]{\lp\DerP{#1}{t}\rp_{conv}}
\newcommand{\tdconvp}[1]{\lp\DerP{#1}{t}\rp_{conv\_\,prec}}
\newcommand{\fsd}{F_{s}^{dif\_\,tur}}
\newcommand{\fqd}{F_{q}^{dif\_\,tur}}
\newcommand{\fhd}{F_{h}^{dif\_\,tur}}
\newcommand{\tsd}{T_{s}^{dif\_\,tur}}
\newcommand{\tqd}{T_{q}^{dif\_\,tur}}
\newcommand{\thd}{T_{h}^{dif\_\,tur}}
\newcommand{\tvasc}{\vec{T}_{\vec{v}}^{conv}}
\newcommand{\omee}{\omega^{*}}
\newcommand{\omec}{\omega_c}
%
%-----------------------------------------------------------------------
% Commande servant à saisir l'essence d'un paragraphe,
% lorsqu'on en est, au cours de la rédaction, encore au stade
% de créer le plan détaillé.
%-----------------------------------------------------------------------
%
\newcommand{\ideepar}[1]{\p[{\bf Paragraphe: }#1]}
%
% Guillemets.
\def\og{\leavevmode\raise.3ex\hbox{$\scriptscriptstyle\langle\!\langle$\kern.05em}}
\def\fg{\leavevmode\unskip\kern.05em\raise.3ex\hbox{$\scriptscriptstyle\rangle\!\rangle$}}
%
% Commande "sujet" ouvrant un nouveau paragraphe séparé du précédent
% par une ligne horizontale et comportant un titre en gras.
\newcommand{\sujet}[1]{\noindent\rule{\textwidth}{0.1mm} {\bf \large #1 \newline }}
%
% Page de titre, 5 arguments:
%   1. Titre.
%   2. Sous-titre.
%   3. Auteur.
%   4. Version et/ou date.
%   5. Nom du fichier graphique à insérer.
%
\newcommand{\entete}[6]{
   {\pagestyle{empty}\null
   \begin{center}
   \begin{tabular}{c}
      \\[1ex] \huge \centerline{#1} \\[1ex]
      \huge #2 \\[2ex]
      \Large #3 \\[2ex] \Large #4 \\[1ex]
   \end{tabular}
   \end{center}
   \null\vspace{1cm}
  \begin{figure}[htbp]
    \centerline{
    \includegraphics
      [angle=0, 
      keepaspectratio=true,
      clip=true,#6]
      {#5}
    }
  \end{figure}
   \newpage\sommaire}}
%
%-----------------------------------------------------------------------
% Graphique commun pour latex et latex2html,
% d'après Ryad El Khatib le 26.5.2003.
%-----------------------------------------------------------------------
%
\newcommand{\graphryad}[5]
  {
  \begin{latexonly}
    \medskip\par
  \end{latexonly}
  \begin{figure}[!h]
    \label{fig:#1}
    \html{\htmlimage{align=center,transparent,antialias}}
    \html{\htmlborder{5}}
    \centering
    \html{\includegraphics[angle=#3]{#1.eps}}
    \latex{\framebox{\includegraphics[scale=#2,angle=#3]{#1.eps}}}
    \begin{latexonly}
      \caption[#4]{#5}
    \end{latexonly}
  \end{figure}
  \html{\begin{center}{\bf Figure~\ref{fig:#1}: }\emph{#5}\end{center}}
  }
%
%-----------------------------------------------------------------------
% Image graphicx en mode figure (ancienne version).
%-----------------------------------------------------------------------
%
\newcommand{\figp}[4]
  {
  \begin{figure}[htbp]
    \centerline{
      \includegraphics
        [angle=#3, 
        width=#2, 
        keepaspectratio=true,
        clip=true]
        {#1}
    }
    \caption{#4}
    \label{#1}
  \end{figure}
  }
%
%-----------------------------------------------------------------------
% Image graphicx en mode figure (nouvelle version).
% [h]: graphique ici.
% [b]: graphique en bas  de page.
% [t]: graphique en haut de page.
% [p]: graphique sur une page à part.
%-----------------------------------------------------------------------
%
\newcommand{\figpn}[6]
  {
  \begin{figure}[htbp]
    \centerline{
      \includegraphics
        [angle=#3, 
        width=#2, 
        keepaspectratio=true,
        clip=true]
        {#1}
    }
    \caption{ {\bf #4} {\em #5} {#6} }
    \label{#1}
  \end{figure}
  }
%
%-----------------------------------------------------------------------
% Image graphicx double. I.e. lit 2 fichiers EPS en entrée,
% et crée une seule figure, avec une seule légende.
% Un graphique est en haut, l'autre en bas.
% figdhb: FIGure Double, Haut-Bas.
%-----------------------------------------------------------------------
%
\newcommand{\figdhb}[7]
  {
  \begin{figure}[htbp]
    \centerline{
      \includegraphics
        [angle=#4, 
        width=#3, 
        keepaspectratio=true,
        clip=true]
        {#1}
    }
    \centerline{
      \includegraphics
        [angle=#4, 
        width=#3, 
        keepaspectratio=true,
        clip=true]
        {#2}
    }
    \caption{ {\bf #5} {\em #6} {#7} }
    \label{#1}
  \end{figure}
  }
%
%-----------------------------------------------------------------------
% Image graphicx double. I.e. lit 2 fichiers EPS en entrée,
% et crée une seule figure, avec une seule légende.
% Les 2 graphiques sont côte-à-côte.
% figdcc: FIGure Double, Côte à Côte.
%-----------------------------------------------------------------------
%
\newcommand{\figdcc}[7]
  {
  \begin{figure}[htbp]
    \centerline{
      \includegraphics
        [angle=#4, 
        width=#3, 
        keepaspectratio=true,
        clip=true]
        {#1}
      \includegraphics
        [angle=#4, 
        width=#3, 
        keepaspectratio=true,
        clip=true]
        {#2}
    }
    \caption{ {\bf #5} {\em #6} {#7} }
    \label{#1}
  \end{figure}
  }
%
%-----------------------------------------------------------------------
% Commande vide "£": elle sert seulement à définir une fonction
% qui ne fait rien sous latex, mais sera reconnue comme
% une chaîne de caractères par le coloriseur du source TEX.
% But: rendre des sources TEX plus lisibles, en pouvant coloriser
% des zones de texte particulières.
%-----------------------------------------------------------------------
%
%\newcommand{\£}{}

\begin{document}
\entete{LFA} {Logiciel de fichiers autodocumentés} {Jean-Marcel Piriou, CNRM/GMAP} 
{\aujour, Version 2} 
{logo_lfa.ps}{height=11.cm,angle=-90.}

\chapter{Présentation générale}

\p  Le présent logiciel permet de lire ou écrire des tableaux de réels,
entiers ou caractères sur des fichiers portables (binaires IEEE), et avec un
code portable. On accède aux champs en les référençant par leur nom.

\p L'idée sous-jacente à ce logiciel est de combiner deux propriétés:
d'une part l'adressage direct des articles par leur nom, pour
un accès plus simple et sécurisé 
aux données du fichier,
et d'autre part l'écriture physique sur des fichiers binaires IEEE
pour garantir la vitesse d'exécution et la portabilité.

\p  L'interface  usager  permet de manipuler
les fichiers depuis des codes fortran,
mais aussi directement depuis la ligne de commande UNIX.


\p On peut ainsi bien sûr
ouvrir/fermer des
fichiers,  écrire/lire des articles,  mais  également  effectuer
des  copies
directes d'un fichier sur l'autre, fusionner
des fichiers, obtenir la liste des
articles, les éléments d'un article (existence, longueur,
type), créer un fichier
LFA depuis la ligne de commande à partir de fichiers texte,
extraire sous forme texte un article donné d'un fichier LFA, 
etc...  
Ceci  permet de libérer les codes de développement de la
"partie basse" de la gestion des fichiers, tout en traitant
une large variété de données.

\section*{Bref historique, lien avec le logiciel LFI}

\p  Ce  logiciel  a  été écrit en octobre 1997. 
Ses fonctionnalités reprennent
celles  du logiciel LFI (Logiciel de Fichiers Indexés) de Jean Clochard,
avec  une interface usager volontairement voisine, mais en y ajoutant 
l'autodocumentation en type et précision (qui permet
notamment la conversion et le traitement automatique
de l'ensemble d'un fichier), et la possibilité de traiter
des articles caractères.

\p  L'interface  usager est voisine pour deux raisons: d'une part, celle
de  LFI  est  suffisamment  agréable et rationnelle pour qu'on n'ait pas
éprouvé le besoin de la modifier, et d'autre part on minimisait ainsi le
temps de portage des codes appelants.

\section*{Performances}

\p Comment se place le logiciel LFA par rapport 
aux autres en termes de vitesse d'exécution et
taille des fichiers?

\p On a mesuré
le temps d'exécution et la taille
des fichiers
pour une écriture puis relecture d'un tableau
de 150000 réels à la précision 7 chiffres significatifs
(\ie\ codés sur 4 octets),
sur une station HP; les temps indiqués sont en secondes CPU
(utilisateur + système), et les tailles en octets.

\p \begin{tabular}{|l|r|r|}\hline 
	Logiciel & Taille & Temps \\ \hline
	non formatté & 600008 & 0.1 \\
	LFA & 600056 & 0.2 \\
	LFI & 605184 & 0.3 \\
	formatté & 2550000 & 15.5 \\
	\hline
\end{tabular}

\p On voit que les formes non formattée, LFI et LFA sont comparables
en taille de fichiers (150000 * 4 octets); en termes de temps calcul, l'écriture
binaire (non formattée, LFA, LFI) est nettement plus favorable que la
transduction en caractères (formattée).

\section*{Précision des données}

\p Le logiciel LFA permet d'écrire/lire des réels
et des entiers sur 4 ou 8 octets;
l'installation du logiciel fournit
les différentes bibliothèques correspondant
aux précisions possibles de la machine
courante.
Par précision on entend ici précision des variables
passées en argument par l'utilisateur au logiciel LFA.
Le nom est explicite, par exemple {\tt liblfa\_R8I4.a} 
pour une bibliothèque
destinée à être appelée par un logiciel utilisateur
à réels sur 8 octets et entiers sur 4.

\p Cependant, le choix d'une précision utilisateur pour
les passages d'arguments ne vous contraint
pas à travailler à cette précision
sur fichier: 
vous pouvez demander à écrire
en 4 octets des tableaux qui ont été calculés sur 8
par exemple (afin de sauver de l'espace disque), ou 
lire sur des tableaux a la précision X des données présentes
sur fichier à la précision Y. Le logiciel LFA fait
l'interface etre les précisions fichier et celles de l'utilisateur
de façon transparente. 

\p Lorsqu'on ne précise
rien de particulier, le défaut est d'écrire à la précision
utilisateur. Cette valeur de précision 
de sortie peut être changée éventuellement
avant chaque article écrit, permettant ainsi
d'écrire au sein d'un même fichier des données de précision
différente (cf lfaprecr et lfapreci).

\section*{Portabilité fichier}

\p Les fichiers seront a priori portables
entre machines ayant la même représentation interne des
données.
Beaucoup de machines respectant aujourd'hui
la norme IEEE, la portabilité des fichiers
est en général possible.

\p Cas du CRAY: cette machine n'est pas IEEE mais peut
produire des fichiers IEEE sur demande, via l'ordre
"assign -N ieee"; cet ordre a donc été écrit 
dans le logiciel LFA, sous clef "\#ifdef cray".

\section*{Synoptique des fonctions}

\centerline{
\includegraphics
	[angle=-90, 
	width=12.cm, 
	keepaspectratio=true,
	clip=true]
	{organigramme.ag.eps}
}


\p Les utilitaires LFA qui font l'objet de la présente
documentation ont été regroupés sur le schéma ci-dessus: l'interface entre
un fichier LFA et la mémoire du calculateur où
va s'exécuter votre programme est effectuée 
par les utilitaires fortran lfalec* (lecture) et lfaecr* (écriture).

\p En rouge sont représentés les utilitaires LFA appelables
DIRECTEMENT depuis la ligne de commande UNIX:
voir la liste des articles d'un fichier, modifier un fichier LFA
à l'aide de son éditeur de texte habituel, extraire un article 
de fichier sur sortie standard, effectuer la différence
de deux fichiers sont des opérations
de routine qui ne nécessiteront donc pas d'écrire un code fortran.

\chapter{Interface fortran usager}

\p On décrit ici les routines que l'utilisateur
sera amené à appeler depuis ses codes fortran
pour lire, écrire ou gérer des fichiers LFA.

\section{lfaouv: ouverture}
 
 
 
\begin{verbatim}
	subroutine lfaouv(kul,cdnomf,cdtypo)
	! -------------------------------------------------------------                
	! **** *LFAOUV* Ouverture de fichier LFA.
	! -------------------------------------------------------------                
	! **** *LFAOUV* Open a LFA file.
	! -------------------------------------------------------------                
	! En entree:
	! kul         unite logique du fichier.
	! cdnomf      nom du fichier.
	! cdtypo      type d'ouverture: 'R' READ, 'W' WRITE, 'A' APPEND, 'S' SCRATCH.
	! En sortie:
	! --------------------------------------------------------------------------
	! Input:
	! kul         logical unit of LFA file.
	! cdnomf      file name.
	! cdtypo      opening type: 'R' READ, 'W' WRITE, 'A' APPEND, 'S' SCRATCH.
	! Output:
	! --------------------------------------------------------------------------
\end{verbatim}
\section{lfafer: fermeture}
 
 
 
\begin{verbatim}
	subroutine lfafer(kul)
	! -------------------------------------------------------------                
	! **** *LFAFER* Fermeture de fichier LFA.
	! -------------------------------------------------------------                
	! **** *LFAFER* Close a LFA file.
	! -------------------------------------------------------------                
	! En entree:
	! kul        unite logique du fichier.
	! En sortie:
	! --------------------------------------------------------------------------
	! Input:
	! kul        logical unit of LFA file.
	! Output:
	! --------------------------------------------------------------------------
\end{verbatim}
 
\section{lfaprecr: précision des réels en écriture}
 
 
\begin{verbatim}
	subroutine lfaprecr(kul,kprec)
	! -------------------------------------------------------------                
	! **** *LFAPRECR* Forcage de la precision d'ecriture des reels.
	! -------------------------------------------------------------                
	! **** *LFAPRECR* Force real data writing precision.
	! -------------------------------------------------------------                
	! En entree:
	! kul      unite logique du fichier LFA.
	! kprec    precision des reels a ecrire ulterieurement, en octets.
	! En sortie:
	! --------------------------------------------------------------
	! Input:
	! kul      logical unit of LFA file.
	! kprec    precision of real data to write, in bytes.
	! Output:
	! --------------------------------------------------------------
\end{verbatim}
 
\section{lfapreci: précision des entiers en écriture}
\begin{verbatim}
	subroutine lfapreci(kul,kprec)
	! -------------------------------------------------------------                
	! **** *LFAPRECI* Forcage de la precision d'ecriture des entiers.
	! -------------------------------------------------------------                
	! **** *LFAPRECI* Force integer data writing precision.
	! -------------------------------------------------------------                
	! En entree:
	! kul      unite logique du fichier LFA.
	! kprec    precision des entiers a ecrire ulterieurement, en octets.
	! En sortie:
	! --------------------------------------------------------------
	! Input:
	! kul      logical unit of LFA file.
	! kprec    precision of integer data to write, in bytes.
	! Output:
	! --------------------------------------------------------------
\end{verbatim}

\section{lfaecrr: écriture de réels}
 
 
 
\begin{verbatim}
	subroutine lfaecrr(kul,cdna,preel,klong)
	! -------------------------------------------------------------                
	! **** *LFAECRR* Ecriture de reels sur fichier LFA.
	! -------------------------------------------------------------                
	! **** *LFAECRR* Write real data on LFA file.
	! -------------------------------------------------------------                
	! En entree:
	! kul              unite logique du fichier.
	! cdna             nom de l'article a ecrire.
	! preel(1,klong)   reels a ecrire.
	! klong            longueur de l'article a ecrire.
	! En sortie:
	! --------------------------------------------------------------------------
	! Input:
	! kul              logical unit of LFA file.
	! cdna             name of article to write.
	! preel(1,klong)   real data to write.
	! klong            length of article to write.
	! Output:
	! --------------------------------------------------------------------------
\end{verbatim}
\section{lfaecri: écriture d'entiers}
 
 
 
\begin{verbatim}
	subroutine lfaecri(kul,cdna,kentier,klong)
	! -------------------------------------------------------------                
	! **** *LFAECRI* Ecriture d'entiers sur fichier LFA.
	! -------------------------------------------------------------                
	! **** *LFAECRI* Write integer data of LFA file.
	! -------------------------------------------------------------                
	! En entree:
	! kul                  unite logique du fichier.
	! cdna                  nom de l'article a ecrire.
	! kentier(1,klong)      entiers a ecrire.
	! klong            longueur de l'article a ecrire.
	! En sortie:
	! --------------------------------------------------------------------------
	! Input:
	! kul                  logical unit of LFA file.
	! cdna                 name of article to write.
	! kentier(1,klong)     integers to write.
	! klong                length of article to write.
	! Output:
	! --------------------------------------------------------------------------
\end{verbatim}
\section{lfaecrc: écriture de caractères}
\begin{verbatim}
	subroutine lfaecrc(kul,cdna,cdcar,klong)
	! -------------------------------------------------------------                
	! **** *LFAECRC* Ecriture de caracteres sur fichier LFA.
	! -------------------------------------------------------------                
	! **** *LFAECRC* Write character data on LFA file.
	! -------------------------------------------------------------                
	! En entree:
	! kul              unite logique du fichier.
	! cdna             nom de l'article a ecrire.
	! cdcar(1,klong)   caracteres a ecrire.
	! klong            longueur de l'article a ecrire.
	! En sortie:
	! --------------------------------------------------------------------------
	! Input:
	! kul              logical unit of LFA file.
	! cdna             name of article to write.
	! cdcar(1,klong)   characters to write.
	! klong            length of article to write.
	! Output:
	! --------------------------------------------------------------------------
\end{verbatim}
\section{lfalecr: lecture de réels}
 
 
 
\begin{verbatim}
	subroutine lfalecr(kul,cdna,kdimb,preel,klong,kerr)
	! -------------------------------------------------------------                
	! **** *LFALECR* Lecture de reels sur fichier LFA.
	! -------------------------------------------------------------                
	! **** *LFALECR* Read real data on LFA file.
	! -------------------------------------------------------------                
	! En entree:
	! kul              unite logique du fichier.
	! cdna             nom de l'article.
	! kdimb            dimension du tableau preel.
	! En sortie:
	! klong            nombre de reels lus.
	! preel(1,klong)   reels lus.
	! kerr             indicateur d'erreur:
	! +----------+-----------------------------------------------------+
	! | Valeur   |             Signification                           |
	! +----------+-----------------------------------------------------+
	! | kerr=  0 | Tout est OK!                                        |
	! | kerr= -1 | Article inexistant                                  |
	! | kerr= -6 | Article plus long que le tableau devant le recevoir |
	! | kerr= -8 | Mauvais type de donnees (reelles, entieres, car.)   |
	! +----------+-----------------------------------------------------+
	! --------------------------------------------------------------------------
	! Input:
	! kul              logical unit of LFA file.
	! cdna             article name.
	! kdimb            physical dimension of array preel.
	! Output:
	! klong            number of real elements read.
	! preel(1,klong)   real elements read.
	! kerr             error indicator:
	! +----------+-----------------------------------------------------+
	! | Value    |             Meaning                                 |
	! +----------+-----------------------------------------------------+
	! | kerr=  0 | Everything is OK!                                   |
	! | kerr= -1 | Article inexistant                                  |
	! | kerr= -6 | Article bigger than array supposed to receive it    |
	! | kerr= -8 | Wrong data type (real, integer, char.)              |
	! +----------+-----------------------------------------------------+
	! --------------------------------------------------------------------------
\end{verbatim}
\section{lfaleci: lecture d'entiers}
 
 
 
\begin{verbatim}
	subroutine lfaleci(kul,cdna,kdimb,kentier,klong,kerr)
	! -------------------------------------------------------------                
	! **** *LFALECI* Lecture d'entiers sur fichier LFA.
	! -------------------------------------------------------------                
	! **** *LFALECI* Read integer data on LFA file.
	! -------------------------------------------------------------                
	! En entree:
	! kul              unite logique du fichier.
	! cdna             nom de l'article.
	! kdimb            dimension du tableau kentier.
	! En sortie:
	! klong            nombre d'entiers lus.
	! kentier(1,klong) entiers lus.
	! kerr             indicateur d'erreur:
	! +----------+-----------------------------------------------------+
	! | Valeur   |             Signification                           |
	! +----------+-----------------------------------------------------+
	! | kerr=  0 | Tout est OK!                                        |
	! | kerr= -1 | Article inexistant                                  |
	! | kerr= -6 | Article plus long que le tableau devant le recevoir |
	! | kerr= -8 | Mauvais type de donnees (reelles, entieres, car.)   |
	! +----------+-----------------------------------------------------+
	! --------------------------------------------------------------------------
	! Input:
	! kul              logical unit of LFA file.
	! cdna             article name.
	! kdimb            physical dimension of array kentier.
	! Output:
	! klong            number of integer elements read.
	! kentier(1,klong) integer elements read.
	! kerr             error indicator:
	! +----------+-----------------------------------------------------+
	! | Value    |             Meaning                                 |
	! +----------+-----------------------------------------------------+
	! | kerr=  0 | Everything is OK!                                   |
	! | kerr= -1 | Article inexistant                                  |
	! | kerr= -6 | Article bigger than array supposed to receive it    |
	! | kerr= -8 | Wrong data type (real, integer, char.)              |
	! +----------+-----------------------------------------------------+
	! --------------------------------------------------------------------------
\end{verbatim}
\section{lfalecc: lecture de caractères}
 
 
\begin{verbatim}
	subroutine lfalecc(kul,cdna,kdimb,cdcar,klong,kerr)
	! -------------------------------------------------------------                
	! **** *LFALECC* Lecture de caracteres sur fichier LFA.
	! -------------------------------------------------------------                
	! **** *LFALECC* Read character data on LFA file.
	! -------------------------------------------------------------                
	! En entree:
	! kul              unite logique du fichier.
	! cdna             nom de l'article.
	! kdimb            dimension du tableau cdcar.
	! En sortie:
	! klong            nombre de chaines de caracteres lues.
	! cdcar(1,klong)   chaines lues.
	! kerr             indicateur d'erreur:
	! +----------+-----------------------------------------------------+
	! | Valeur   |             Signification                           |
	! +----------+-----------------------------------------------------+
	! | kerr=  0 | Tout est OK!                                        |
	! | kerr= -1 | Article inexistant                                  |
	! | kerr= -6 | Article plus long que le tableau devant le recevoir |
	! | kerr= -8 | Mauvais type de donnees (reelles, entieres, car.)   |
	! +----------+-----------------------------------------------------+
	! --------------------------------------------------------------------------
	! Input:
	! kul              logical unit of LFA file.
	! cdna             article name.
	! kdimb            physical dimension of array cdcar.
	! Output:
	! klong            number of character elements read.
	! cdcar(1,klong)   character elements read.
	! kerr             error indicator:
	! +----------+-----------------------------------------------------+
	! | Value    |             Meaning                                 |
	! +----------+-----------------------------------------------------+
	! | kerr=  0 | Everything is OK!                                   |
	! | kerr= -1 | Article inexistant                                  |
	! | kerr= -6 | Article bigger than array supposed to receive it    |
	! | kerr= -8 | Wrong data type (real, integer, char.)              |
	! +----------+-----------------------------------------------------+
	! --------------------------------------------------------------------------
\end{verbatim}
\section{lfatest: le fichier est-il LFA?}
 
 
 
\begin{verbatim}
	subroutine lfatest(kul,cdnomf,ldlfa)
	! -------------------------------------------------------------                
	! **** *LFATEST* Teste si un fichier est bien de type LFA.
	! -------------------------------------------------------------                
	! **** *LFATEST* Test if a file is a LFA one.
	! -------------------------------------------------------------                
	! En entree:
	! kul         unite logique du fichier;
	! .           ce doit etre une unite disponible:
	! .           le fichier va etre ouvert sous cette unite logique.
	! cdnomf      nom du fichier.
	! En sortie:
	! ldlfa=.true. si le fichier est de type LFA, .false. sinon.
	! --------------------------------------------------------------------------
	! Input:
	! kul         logical unit of file.
	! .           this unit has to be free:
	! .           the file will be opened with this logical unit.
	! cdnomf      file name.
	! Output:
	! ldlfa=.true. if the file is a LFA one, .false. else case.
	! --------------------------------------------------------------------------
\end{verbatim}
 
\section{lfames: niveau de messagerie}
 
 
 
\begin{verbatim}
	subroutine lfames(kul,kmes)
	! -------------------------------------------------------------                
	! **** *LFAMES* Niveau de messagerie du logiciel LFA.
	! -------------------------------------------------------------                
	! **** *LFAMES* Print out level of LFA software.
	! -------------------------------------------------------------                
	! En entree:
	! kul         unite logique du fichier.
	! kmes        niveau de messagerie:
	! si 0 aucun message sorti par le logiciel LFA.
	! si 1 messages d'ATTENTION et d'ERREUR sorties.
	! si 2 LFA est bavard (a reserver au debug de LFA...).
	! En sortie:
	! --------------------------------------------------------------------------
	! Input:
	! kul         logical unit of LFA file.
	! kmes        print out level:
	! if 0 no message print out.
	! if 1 WARNING or ERROR messages print out.
	! if 2 many comments print out (LFA debug mode only...).
	! Output:
	! --------------------------------------------------------------------------
\end{verbatim}
 
\section{lfaerf: niveau d'erreur admis}
 
 
 
\begin{verbatim}
	subroutine lfaerf(kul,lderf)
	! -------------------------------------------------------------                
	! **** *LFAERF* Niveau d'erreur tolere par le logiciel LFA.
	! -------------------------------------------------------------                
	! **** *LFAERF* Choose error level of LFA software.
	! -------------------------------------------------------------                
	! En entree:
	! kul               unite logique du fichier.
	! lderf             .true. si toute erreur doit etre fatale,
	! .false. si aucune ne doit l'etre.
	! En sortie:
	! lgerf             .true. si toute erreur est fatale,
	! .false. si aucune ne l'est.
	! --------------------------------------------------------------------------
	! Input:
	! kul               logical unit of LFA file.
	! lderf             .true. if any error has to be fatal.
	! .false. si none has to be.
	! Output:
	! lgerf             .true. if any error has to be fatal.
	! .false. si none has to be.
	! --------------------------------------------------------------------------
\end{verbatim}
\section{lfalaf: liste des articles sur output standard}
 
 
 
\begin{verbatim}
	subroutine lfalaf(kul,kulout)
	! -------------------------------------------------------------                
	! **** *LFALAF* Liste des articles d'un fichier LFA.
	! -------------------------------------------------------------                
	! **** *LFALAF* Article list of a LFA file.
	! -------------------------------------------------------------                
	! En entree:
	! kul             unite logique du fichier.
	! kulout          unite logique sur laquelle sortir la liste.
	! En sortie:
	! --------------------------------------------------------------------------
	! Input:
	! kul             logical unit of LFA file.
	! kulout          logical unit on which print out the list.
	! Output:
	! --------------------------------------------------------------------------
\end{verbatim}
\section{lfalaft: liste des articles sur tableau de caractères}
 
 
 
\begin{verbatim}
	subroutine lfalaft(kul,cdlis,kdlis,knlis)
	! -------------------------------------------------------------                
	! **** *LFALAFT* Liste des articles d'un fichier LFA sur tableau de caracteres.
	! -------------------------------------------------------------                
	! **** *LFALAFT* Article list of a LFA file, on an array.
	! -------------------------------------------------------------                
	! En entree:
	! kul             unite logique du fichier.
	! kdlis           dimension physique du tableau cdlis.
	! En sortie:
	! knlis           nombre d'articles du fichier.
	!                 Ce nombre est egalement le nombre d'elements ecrits sur cdlis
	! cdlis(1, ..., knlis) nom des articles du fichier.
	! --------------------------------------------------------------------------
	! Input:
	! kul            logical unit of LFA file.
	! kdlis          physical dimension of array cdlis.
	! Output:
	! knlis          number of articles on the file.
	!                This number is also the number of elements written on cdlis.
	! cdlis(1, ..., knlis) article names.
	! --------------------------------------------------------------------------
\end{verbatim}
 
\section{lfaminm: extrema de tous les articles}
 
 
\begin{verbatim}
	subroutine lfaminm(kul)
	! -------------------------------------------------------------                
	! **** *LFAMINM* Extrema de tous les articles d'un fichier LFA.
	! -------------------------------------------------------------                
	! **** *LFAMINM* Extrema of all articles of a given LFA file.
	! -------------------------------------------------------------                
	! En entree:
	! kul unite logique du fichier LFA d'entree.
	! En sortie:
	! Extrema sur output standard.
	! --------------------------------------------------------------
	! Input:
	! kul logical unit of LFA file.
	! Output:
	! Extrema on standard output.
	! --------------------------------------------------------------
\end{verbatim}
\section{lfacas: renseignements sur un article}
\begin{verbatim}
	subroutine lfacas(kul,cdna,cdtype,klong,kerr)
	! -------------------------------------------------------------                
	! **** *LFACAS* Renseignements sur un article de fichier LFA.
	! -------------------------------------------------------------                
	! **** *LFACAS* Get documentation about a LFA article.
	! -------------------------------------------------------------                
	! En entree:
	! kul               unite logique du fichier.
	! cdna: si cdna=' ' on recherche l'article suivant.
	! .         cdna est alors en entree/sortie,
	! .         et en sortie il vaudra le nom de l'article suivant
	! .         (si cet article existe).
	! .         kerr...retour de recherche: 0 si OK,
	! .                1 si fin de fichier.
	! .     si cdna<>' ' cdna est le nom de l'article cherche.
	! .          Il est alors en entree seulement.
	! .         kerr...retour de recherche: 0 si OK,
	! .                1 si article inexistant.
	! En sortie:
	! cdtype            type d'article: 'R4', 'I8', 'C '.
	! klong             nombre d'elements de cet article.
	! --------------------------------------------------------------------------
	! Input:
	! kul               file logical unit.
	! cdna: if cdna=' ' on looks for nbext article.
	! .         cdna is then in input/output
	! .         and in output it will receive next article name
	! .         (if this article exists).
	! .         kerr...return from search: 0 if OK,
	! .                1 if end of file.
	! .     if cdna<>' ' cdna is the name from required article.
	! .          It is then in input olny.
	! .         kerr...return from search: 0 if OK,
	! .                1 if non-existant article.
	! Output:
	! cdtype            article type: 'R4', 'I8', 'C '.
	! klong             numbre of elements in this article.
	! --------------------------------------------------------------------------
\end{verbatim}
\section{lfaavan: saut de l'article courant}
 
 
\begin{verbatim}
	subroutine lfaavan(kul)
	! -------------------------------------------------------------                
	! **** *LFAAVAN* Saut de l'article courant dans un fichier LFA.
	! -------------------------------------------------------------                
	! **** *LFAAVAN* Step over current article in an LFA file.
	! -------------------------------------------------------------                
	! En entree:
	! kul               unite logique du fichier.
	! En sortie:
	! --------------------------------------------------------------------------
	! Input:
	! kul               logical unit of the LFA file.
	! Output:
	! --------------------------------------------------------------------------
\end{verbatim}

\section{lfarew: rebobinage du fichier}

\p Cet appel sert dans le cas rare suivant: vous avez lu
certains articles du fichier, puis vous voulez lire tous les articles
du fichier séquentiellement via lfacas. lfacas fournissant 
le nom de l'article suivant, il faut au préalable rebobiner
le fichier par lfarew.

\p Ce cas est rare: en général, soit on lit des articles en y accédant
directement par leur nom, auquel cas la gestion
du pointeur fichier est effectuée de façon 
transparente par le logiciel LFA, soit on veut lire tout le fichier
séquentiellement, et on le fait dès son ouverture,  et il
n'y a donc pas lieu de rebobiner!... 
 
 
 
\begin{verbatim}
	subroutine lfarew(kul)
	! -------------------------------------------------------------                
	! **** *LFAREW* Rebobinage d'un fichier LFA.
	! -------------------------------------------------------------                
	! **** *LFAREW* Rewind a LFA file.
	! -------------------------------------------------------------                
	! En entree:
	! kul: unite logique du fichier LFA.
	! En sortie:
	! --------------------------------------------------------------
	! Input:
	! kul: logical unit of LFA file.
	! En sortie:
	! --------------------------------------------------------------
\end{verbatim}

\section{lfacop: copie d'article de fichier à fichier}
\begin{verbatim}
	subroutine lfacop(kule,cdnae,cdnas,kuls)
	! -------------------------------------------------------------                
	! **** *LFACOP* Copie d'un article d'un fichier LFA a un autre.
	! -------------------------------------------------------------                
	! **** *LFACOP* Copy one article from a LFA file to another.
	! -------------------------------------------------------------                
	! En entree:
	! kule unite logique du fichier LFA d'entree.
	! cdnae nom de l'article a lire.
	! cdnas nom sous lequel l'article est recopie.
	! kuls unite logique du fichier LFA de sortie.
	! En sortie:
	! Le fichier d'unite logique kuls est augmente d'un article.
	! --------------------------------------------------------------
	! Input:
	! kule logical unit of input LFA file.
	! cdnae article name to be read.
	! cdnas article name to be written out.
	! kuls logical unit of output LFA file.
	! Output:
	! The file which logical unit is kuls receives one more article.
	! --------------------------------------------------------------
\end{verbatim}

\chapter{Interface UNIX usager}

\p  Les fichiers LFA pouvant contenir
ds données très diverses, 
il est utile, lorsqu'on  dispose  de fichiers LFA, 
d'effectuer  sur  ces  fichiers  un  certain  nombre d'opérations simples
directement  depuis  la  ligne  de  commande  du système d'exploitation:
savoir la liste des articles qu'ils contiennent, quels sont les extrêmes
de  tel article, pouvoir créer un
fichier  "à  la  volée" a partir de $n$ fichiers textes, etc... Afin que
chacun  ne  soit pas amené à récrire ces utilitaires, leurs sources sont
mis  à  disposition  avec la bibliothèque LFA, et leurs exécutables sont
générés lors de l'installation du logiciel.

\p Leurs descriptions synoptiques et modes d'emploi sont proposés
ci-dessous. Ces élements peuvent être obtenus également depuis
la ligne de commande: taper une commande LFA (par exemple: lfalaf)
sans argument provoque la sortie à l'écran du synoptique de la commande.

\section{lfalaf: liste des articles}

\begin{verbatim}
  
 Sortie sur output standard de la liste des articles d'un fichier lfa.
  
 Utilisation: lfalaf nomf

\end{verbatim}

\section{lfaminm: sortie des extrema et moyenne}
\begin{verbatim}
 
Sortie des extrema et moyenne des articles d'un (plusieurs) fichier(s) LFA.
 
Utilisation: lfaminm LFA1 [LFA2 ... LFAn]
 
\end{verbatim}

\section{lfaedit: édition sous forme texte}

\begin{verbatim}
 
Edition d'un(plusieurs) fichier(s) LFA.
Le but est ici de pouvoir visualiser ou modifier à l'aide
de son éditeur habituel toutes les valeurs d'un fichier LFA.
 
Utilisation: lfaedit F1 [F2 ... Fn]
 
Principe: les fichiers sont transformés en leur forme LFP (ASCII texte),
puis on appelle l'éditeur. Les fichiers en sortie de l'éditeur,
s'ils ont été modifiés, sont retransformés en leur forme LFA.
L'éditeur appelé est celui défini par la variable
d'environnement EDITOR.

\end{verbatim}


\section{lfac: extraction d'un article}
\begin{verbatim}
  
 Extraction sur output standard d'un article de fichier lfa.
  
 Utilisation: lfac nomf nomc 
 avec
     nomf nom du fichier lfa d'entree.
     nomc nom du champ a extraire.
  
\end{verbatim}

\section{lfacop: copie d'articles d'un fichier à l'autre}
\begin{verbatim}
   
 Copie de n articles d'un fichier LFA sur un autre fichier LFA.
   
 Utilisation: lfacop LFA1 LFA2 ART1 [ART2 ... ARTn]
 avec
     LFA1 nom du fichier LFA d'entree.
     LFA2 nom du fichier LFA de sortie.
     ART1 [ART2 ... ARTn] la liste des articles à copier.
   
\end{verbatim}

\section{lfareu: réunion de deux fichiers}
\begin{verbatim}
  
 Reunion de deux fichiers lfa.
  
 Utilisation: lfareu F1 F2 Fres
        En entree: F1 et F2, en sortie: Fres.
        F2 est prioritaire sur F1, i.e. si un article
        est present dans F1 et F2, c'est celui de F2 qui sera copie.
  
\end{verbatim}

\section{lfamoy: moyenne de n fichiers}
\begin{verbatim}
  
 Moyenne de n fichiers LFA.
  
 Utilisation: lfamoy FMOY F1 F2 [F3 ... Fn]
 avec
        F1 F2 [F3 ... Fn] les n fichiers d'entrée.
        FMOY le fichier de sortie, recevant la moyenne.
  
 Remarque: la moyenne est opérée sur les articles
 communs aux n fichiers.
\end{verbatim}

\section{lfacre: création d'un fichier depuis la ligne de commande}
\begin{verbatim}
 
Création d'un fichier lfa à partir de la ligne de commande
et(ou) de fichier(s) texte.
 
Utilisation:
 
lfacre LFA [nom_1 type_1 valfic_1] ... [nom_n type_n valfic_n]
n doit valoir au plus  20
En sortie, le fichier LFA contiendra les n articles
nom_1 à nom_n, dont le type sera type_1 à type_n (type: R, I, ou C),
et dont le contenu sera valfic_1 à valfic_n:
        - Si valfic_i est un fichier, alors le contenu
          de ce fichier sera le contenu de l'article nom_i.
        - Si valfic_i n'est pas un fichier, alors c'est la valeur
          du contenu de l'article de longueur 1 nom_i.
 
Exemple:
 
cat <<EOF > gol
gol1
gol2
EOF
lfacre LFA I0  R 1370. indice C gol i8 I 8
créera le fichier LFA, contenant trois articles, l'article réel I0
(longueur 1), l'article caractère indice (longueur 2),
et l'article entier i8 (longueur 1).

\end{verbatim}
\section{lfadiff: différence de deux fichiers}
\begin{verbatim}
  
 Différence de deux fichiers lfa.
  
 Utilisation: lfadiff F1 F2 FDIFF
 avec
        F1 et F2 les deux fichiers d'entrée.
        FDIFF le fichier de sortie, recevant F1-F2.
  
 Remarque: la différence est opérée sur les articles
 communs aux deux fichiers.
  
\end{verbatim}
\section{lfadiffrel: différence relative de deux fichiers}
\begin{verbatim}
 
 Différence relative de deux fichiers LFA.
  
 Utilisation: lfadiffrel F1 F2 FDIFF
 avec
        F1 et F2 les deux fichiers d'entrée.
        FDIFF le fichier de sortie, recevant (F2-F1)/rcm(F1).
  
 rcm(F1) est la racine du carré moyen de l'article de F1.
 Remarque: la différence est opérée sur les articles
 communs aux deux fichiers.
  
 Lorsque rcm(F1)=0, le résultat est nul si F2=0, et égal à  999.999
  sinon.
  

\end{verbatim}
\section{lfadiffart: différence d'articles entre deux fichiers}
\begin{verbatim}
  
 Différence entre les listes d'articles de deux fichiers LFA.
  
 Utilisation: lfadiffart F1 F2 
 avec
        F1 et F2 les deux fichiers d'entrée.
  
\end{verbatim}

\section{lfa2lfp: obtention d'une version ASCII texte du fichier}

\begin{verbatim}
  
 Conversion d'un fichier LFA en un fichier LFP.
  
 Sujet:
  
 Les fichiers binaires ne sont lisibles que par un logiciel.
 Or il serait souvent utile de naviguer directement dans les
 données à l'aide d'un simple éditeur, pour voir
 les valeurs individuelles, pouvoir les imprimer, etc...
 Le présent convertisseur transforme un fichier LFA (binaire IEEE)
 en un fichier ASCII texte comportant in extenso les noms d'articles,
 leur taille, type, et tous leurs éléments. Un tel fichier
 peut egalement transiter par le courrier électronique.
  
 Utilisation: lfa2lfp nom_fic_entree nom_fic_sortie
\end{verbatim}

\section{lfp2lfa: passage d'une version ASCII texte à un LFA}

\p Conversion inverse de la précédente.

\begin{verbatim}
  
 Conversion d'un fichier LFP en un fichier LFA.
  
 Utilisation: lfp2lfa nom_fic_entree nom_fic_sortie

\end{verbatim}

\section{lfa2lfa: modification de la précision des données}

\begin{verbatim}
  
 Conversion d'un fichier LFA en un autre fichier LFA,
 en forçant la précision des réels et des entiers.
  
 Utilisation: lfa2lfa [-i] [-r] nom_fic_entree nom_fic_sortie
  
 avec
        -i8 si on veut en sortie des entiers sur 8 octets.
        -i4 si on veut en sortie des entiers sur 4 octets.
            défaut:  4
        -r8 si on veut en sortie des réels   sur 8 octets.
        -r4 si on veut en sortie des réels   sur 4 octets.
            défaut:  4
 
 Exemple:
        lfa2lfa -r8 -i4 LFA LFARES

\end{verbatim}

Cet utilitaire permet par exemple de gagner de la place
fichier en convertissant a posteriori en 4 octets des données inutilement
écrites sur 8.

\chapter{Exemples d'utilisation}
%%%%%%%%%%%%%%%%%%%%%%%%%%%%%%%%%%%%%%%%%%%%%%%%%%%%%%%%%%%%%%%%%%%%%%%%%%%%%%
\section{Simple écriture / lecture}

\p La routine LFAPPDEMO du source lfa.F
écrit sur un fichier LFA des
données entières, réelles et caractères,
sort la liste des articles écrits, puis relit ces données.
%%%%%%%%%%%%%%%%%%%%%%%%%%%%%%%%%%%%%%%%%%%%%%%%%%%%%%%%%%%%%%%%%%%%%%%%%%%%%%
\section{Lecture séquentielle de tout un fichier}

\p Un des intérêts du logiciel LFA est que les articles sont autodocumentés
en nom, taille et type de sorte qu'il est possible de lire
tout un fichier sans a priori sur son contenu. Exemple avec le programme
lecture\_sequentielle.f
qui donne les moyenne et écart-type de tous les articles réels 
d'un fichier LFA donné.


\section{Lecture hors du cadre LFA}

\p Lorsqu'on fournit de façon ponctuelle un fichier LFA
à quelqu'un ne disposant pas du logiciel -et ne souhaitant
pas l'installer pour lire un seul fichier!-, on joint
avec le logiciel le programme lecture\_directe\_lfa.f,
qui saute les articles
d'autodocumentation pour fournir les seuls
articles de données réelles et entières.

\chapter{Présentation détaillée: principes de fonctionnement interne}

%%%%%%%%%%%%%%%%%%%%%%%%%%%%%%%%%%%%%%%%%%%%%%%%%%%%%%%%%%%%%%%%%%%%%%%%%%%%%%
\section{Structure des fichiers}

\p Les fichiers en entrée/sortie sont des fichiers séquentiels 
non formattés, respectant la norme IEEE; leur structure est la suivante:

\begin{enumerate}
	\item En-tête du fichier: on écrit un entier fixe. Cet entier
		permet au logiciel LFA de vérifier lors de chaque
		ouverture en lecture d'un fichier,
		que celui-ci avait bien auparavant été écrit
		via le logiciel LFA. La valeur
		de cet entier permet également de voir la version
		du logiciel qui a écrit le fichier, afin
		d'assurer la compatibilité ascendante lors
		de futures évolutions.

		\begin{verbatim}
integer*4 iversion
		\end{verbatim}
	\item On rencontre ensuite, autant de fois que d'articles:
		\begin{description}
		    \item [Une autodocumentation:] c'est une ligne d'entiers,
		    	contenant successivement le type de la donnée
			(réel, entier ou caractère), sa longueur
			(logique, pas la longueur physique en octets),
			la longueur L en caractères du nom de l'article,
			puis la suite des L codes ASCII des caractères
			du nom de l'article.
			L'écriture fortran est du type: 
			\newline {\tt write(kul) ktype,klong,ilna,(ichar(cdna(jna:jna)),jna=1,ilna)}.
			avec
		\begin{verbatim}
integer*4 ktype,klong,ilna,ichar
		\end{verbatim}
		    \item [Des données:] il s'agit d'un write implicite de 
		    	réels, d'entiers ou de caractères. Les réels et entiers
			peuvent être sur 4 ou 8 octets, ceci étant
			autodocumenté par la valeur de {\tt ktype} ci-dessus.
		\end{description}
\end{enumerate}

\section{Position du pointeur}

\p Le tableau lgpoint du common lfacom contient la position du pointeur
du  fichier  séquentiel,  sous  la  forme  suivante:  s'il  est vrai, le
pointeur  est  avant  un  article  de  données, et sinon il est avant un
article  d'autodocumentation. C'est à partir de cette information que le
logiciel  cherche  un  article  demandé  à  partir  de  n'importe quelle
position du pointeur dans le fichier. Cette information est {\em interne}
au logiciel, elle n'a pas à être connue de l'utilisateur,
qui gère lui de façon transparente un appel à un article par son nom.

\chapter{Versions du logiciel}

\section*{Version 1 (1997-10)}

\p Version disposant
des aspects autodocumentation,
gestion transparente des aspects physiques du fichier,
utilitaires appelables depuis le système d'exploitation,
etc...

\section*{Version 1.1 (1997-11, iversion=8)}

\p Les chaînes
de caractères sont maintenant écrites de façon implicite,
alors qu'avant on écrivait la suite des entiers des codes
ASCII des caractères. On gagne donc un facteur 4 sur 
la taille des articles caractères. La compatibilité
en lecture des anciens fichiers est assurée.

\section*{Version 1.2 (1998-02, iversion=9)}

\begin{itemize}
	\item Correction d'une petite bug
		concernant l'impression sur output standard en cas
		d'article non trouvé dans un fichier LFA: la longueur
		du nom d'article y était parfois erronée.
	\item Correction d'une bug active dans le seul cas de la machine
		Cray, et dans le cas suivant:
		vous ouvrez 
		un fichier avec le logiciel LFA, et ce fichier n'est pas un LFA;
		vous obtenez alors un code réponse d'erreur. Vous 
		ouvrez alors le fichier avec un autre logiciel. Le logiciel
		LFA n'ayant pas exécuté l'assign inverse (pour supprimer
		la transduction FORMAT\_IEEE/FORMAT\_CRAY), les programmes
		fortran suivants vont être affectés par la transduction.
		La correction de bug a consisté ici à effectuer
		l'ordre assign inverse dans tous les cas de figure du logiciel
		LFA.
	\item Enrichissement des codes de réponse d'erreur.
\end{itemize}
\section*{Version 2.0 (1998-03, iversion=9)}

\begin{itemize}
	\item Il est souvent utile d'écrire ou lire
		des réels ou entiers à une précision
		différente de celle a laquelle
		ils sont calculés par le programme utilisateur.
		C'est ce que permet la version 2 de LFA:
		le logiciel permet maintenant de lire
		et d'écrire des données réelles ou entières en
		32 ou 64 bits, suivant le choix
		de l'utilisateur, et ce quel que soit la taille
		des tableaux utilisateur passés en argument au logiciel.
		On peut ainsi calculer en 64 bits puis
		passer le tableau 64 bits au logiciel
		LFA qui l'écrira en 32 ou 64 bits suivant
		le choix que vous aurez fait via lfaprecr ou lfapreci.
		L'appel à lfaprecr ou lfapreci peut être répété
		plusieurs fois, de sorte que que l'on peut successivement au sein 
		d'un même fichier écrire des données en 32 et 64 bits.
		La taille est autodocumentée dans le fichier
		de sorte que la relecture s'effectuera de façon transparente.
		Il sera ainsi transparent de lire des fichiers LFA
		ayant été écrits sur des machines IEEE différentes,
		avec des précisions différentes.
	\item La lecture des anciens fichiers
		par le nouveau logiciel est possible s'ils ont
		été écrits en entiers 32 bits, et réels 32 bits.
	\item La lecture des nouveaux fichiers par les anciennes
		versions du logiciel LFA est impossible.
	\item La conservation par transitivité de la précision
		à laquelle chaque article du fichier a été écrit
		est assurée par les différents utilitaires UNIX lfaedit,
		lfareu, etc... Ce qui veut dire que les manipulations
		que vous effectuez via les unitaires UNIX de LFA 
		conserveront la précision que vous aviez initialement
		choisie pour vos données. L'utilitaire lfa2lfa
		permet, lui, de modifier cette précision en précisant
		celle voulue sur le fichier de sortie.
	\item L'installation  a  été simplifiée, puisqu'aucun paramètre n'est
		plus  nécessaire  pour  installer  le  logiciel:  le type de machine est
		reconnu  par  le  script  d'installation, qui crée alors les différentes
		librairies  associées aux précisions entières ou réelles disponibles sur
		ladite machine,     
		avec un nom explicite, par exemple pour une bibliothèque
		LFA destinée à être appelée par un logiciel utilisateur
		à réels sur 8 octets et entiers sur 4: {\tt liblfa\_R8I4.a}.
	\item Le logiciel devient bilingue, en affichant ses messages
		en français ou anglais suivant la valeur de la variable d'environnement 
		LANG: si LANG a ses deux premiers caractères égaux à "fr",
		LFA affiche en français; il affiche en anglais dans les autres
		cas.
	\item Ajout de l'utilitaire lfadiffart.
\end{itemize}

\section*{Version 2.1 (1998-11, iversion=21)}

\p Modification de l'interprétation
de l'autodocumentation des fichiers LFA par le logiciel:
avant la version
LFA 2.0, le code 1 pouvait correspondre à des réels
4 ou 8 octets, ceci suivant les options
avec lesquelles LFA avait été compilé.
A partir de la version 2.0, 1 correspond aux
réels 8 octets, et 3 aux réels 4 octets;
le fichier est donc mieux autodocumenté,
puisque LFA sait reconnaître la précision sur fichier
et l'interface de façon transparente avec celle 
du tableau de l'utilisateur.

\p Cependant il est clair que les versions même
récentes de LFA ne peuvent
pas "deviner" la précision des réels 
écrite dans les fichiers antérieurs à LFA 2.0,
puisque cette information était absente.
On est donc obligé de prendre une décision arbitraire.
A partir de LFA 2.1, cette précision arbitraire
des réels lus sur les fichiers écrits
par les versions antérieures à LFA 2.0 est supposée égale à 8 octets
(si on a compilé avec {\tt r4pre2})
ou à 4 octets (sinon). Elle peut donc être réglée lors de la
{\em compilation}.

\end{document}
