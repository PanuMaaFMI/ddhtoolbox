\chapter{Main principles}

\null
\vspace{1cm}
 
This paper presents the diagnostics on horizontal domains (DDH) developed initally for
the variable mesh of \ARP (global stretched model), and now also available for
\ALA\ (Limited Area Model) and \ARO\ (Cloud Systems Resolving Model).

The main objective of the DDH tool is to provide, on user defined domains, the budget
of prognostic variables of these models (momentum, temperature, water vapour,
etc). The DDH tool is used by searchers and model developpers to understand the
model's dynamical and physical interactions, thus contributing to the
parameterization development process. The DDH are also
used for other purposes, like getting mean model drifts with respect to analyses,
or to extract model vertical profiles at given locations.

These diagnostics are made, on the one side, of logical functions which enable
to manage several user defined horizontal domains depicting parts or the whole
of the domain of integration, while reducing the number of scientific
computations. On the other hand, they include the production of budget equations
of the prognostic variables for domains such as zonal bands, rectangular areas, single
vertical model columns, or the whole globe.

Each model point is described within the DDH sofware by its geographical
position, a scale factor, the orientation of the geographical North vector, a
mean value of each variable as well as some horizontal derivatives.

Each point is independent and can belong to different user-defined domains; any
mean on several points makes a domain. The gathering of all points makes the
global domain. A zonal mean is made on a grouping of points between two given
latitudes, etc.

DDH therefore makes a double representation of domains: a first one, external,
which operates groupings meaningful for users; that is a zonal band, or points
within a given geographical area. The other one, internal, consists in the
grouping of any other points, not even necessarily connex ones.

Information within a given domain is also categorized by DDH: for some
parameters, an instantaneous value is required, for others, only a mean value in
time is required. This implies the use of operators of horizontal means and of
simple and linear mean values in time, commutative as often as possible.

\section{User-defined horizontal domains}

\begin{itemize}
	\item whole globe,
	\item zonal bands of equal surfaces,
	\item limited domains, defined by either two or four corners, overlapping or not,
	\item isolated points, either inside or outside the above mentioned.
\end{itemize}

In practice, for the DDH software, limited domains and isolated points are the same kind.
\section {User-defined outputs} 
\begin {itemize}
	\item a print on standard output of budgets on any given domain, in vertical mean.
	\item the production of a file such as {\tt LFA} which includes vertical profiles
	of mean horizontal parameters, eventually cumulated in time, on this
	ensemble of domains.
	Thus, the following files can be produced: 
	\begin{itemize}
		\item a file for the whole model domain,
		\item a file containing zonal bands,
		\item a file of limited domains and of isolated points,
	\end{itemize}
\end{itemize}

\section{Scientific content of diagnostics}
\begin{itemize}
	\item Budgets of mass / energy: balance of air and water masses, potential
	and internal energies, kinetic energy, momentum budget, entropy budget,
	can be activated from independent logical indicators.
	\item For surface, the DDH tool provides surface fluxes,
		but not soil budgets, as these soil budgets are given
		by the SURFEX software itself.
\end{itemize}

\section{Output frequency} 

For DDH diagnostics, two independent control chains are available. One for
printing, the other one for producing files. They enable to choose a time unit
(time step or prediction time) and to generate output either at regular
intervals or at specific irregular prediction times.

\section{Internal representation of domains}
The main principle of DDH is to allocate \underbar{each model point} to \underbar{one and only one internal domain}, and this even if this point belongs to several {\em user} domains. With this principle, scientific computations are made only once, thus saving computation time. Therefore, two ways to split the integration domain are used:
\begin{itemize}
	\item \underbar{the external (or user-defined) splitting}: a given point can be at the same time, within the global domain, within a zonal band, and within one or more limited domains. Of course, final file or listing outputs are presented in this user-defined geometry, but one should try to make the least possible computations. 
	\item \underbar{the internal splitting}, meaning the partition (in the sense of
		the ensemble theory) of the integration domain. Therefore, to each point is
		associated a single internal domain. The associated mask is set in the
		same way as the geographical positions' one, land/sea indicator, etc. To each
		user's domain corresponds a single set of internal domains. This set entirely
		defines the external domain.
\end{itemize}
\section{Horizontal mean operator}

It must be such that the average of the mean values of each sub ensemble gives
the global mean value on the sphere (or on the entire model domain, in the case
of a limited area model). Each point is given a weight which represents its
geographical surface (but without dimension). 

